%% magick convert -density 1200 test.pdf test.png
\documentclass[preview, border={0pt 2pt 1pt 1pt}, varwidth=16.3cm]{standalone} % border options are {left bottom right top}

% \usepackage[T1]{fontenc}
% \usepackage[sfdefault]{AlegreyaSans}
% \renewcommand*\oldstylenums[1]{{\AlegreyaSansOsF #1}}

% \usepackage[sfdefault]{FiraSans} %% option 'sfdefault' activates Fira Sans as the default text font
% \usepackage[T1]{fontenc}
% \renewcommand*\oldstylenums[1]{{\firaoldstyle #1}}

\usepackage[T1]{fontenc}
\usepackage{Alegreya} %% Option 'black' gives heavier bold face 
\renewcommand*\oldstylenums[1]{{\AlegreyaOsF #1}}


% \usepackage[default]{lato}
% \usepackage[T1]{fontenc}

% \usepackage[sfdefault]{cabin}
% \usepackage[T1]{fontenc}

% \usepackage[sfdefault]{roboto}  %% Option 'sfdefault' only if the base font of the document is to be sans serif
% \usepackage[T1]{fontenc}

% \usepackage[sfdefault]{noto}
% \usepackage[T1]{fontenc}

% \usepackage[T1]{fontenc}
% \usepackage[sfdefault]{josefin}

\usepackage{amsthm}
\usepackage{amssymb}
\usepackage{amsmath}
\usepackage{bm}
\usepackage{nicefrac}
\usepackage{graphicx}
\usepackage{tikz}
\usepackage{booktabs}
\usepackage{soul}

\usetikzlibrary{automata, positioning}

%% Colors
% grey
\definecolor{DarkCharcoal}{RGB}{30, 30, 30} % theorem
\definecolor{Davys}{RGB}{60, 60, 60} % proof
\definecolor{SonicSilver}{RGB}{90, 90, 90} % proposition
\definecolor{Gray}{RGB}{120, 120, 120} % lemma
\definecolor{Spanish}{RGB}{150, 150, 150} % corollary
\definecolor{Argent}{RGB}{180, 180, 180} % example

% blues
\definecolor{Azure}{RGB}{0, 128, 255}
\definecolor{Egyptian}{RGB}{16, 52, 166}
\definecolor{Navy}{RGB}{17, 30, 108}
\definecolor{CeruleanFrost}{RGB}{104, 145, 195}
\definecolor{LightSteelBlue}{RGB}{170, 197, 226}

% reds
\definecolor{Renate}{RGB}{227, 25, 113}
\definecolor{Burgundy}{RGB}{141, 2, 31}
\definecolor{Salmon}{RGB}{250, 128, 114}

% yellows
\definecolor{Honey}{RGB}{235, 150, 5}

% greens
\definecolor{PastelGreen}{HTML}{6ECB63}
\definecolor{FernGreen}{HTML}{116530}
\definecolor{CadmiumGreen}{HTML}{046C41}


%% Color Schemes
% Pastel Tones Color Scheme
\definecolor{Manatee}{RGB}{154, 145, 172} % darkest
\definecolor{Lilac}{RGB}{202, 167, 189}
\definecolor{CherryBlossomPink}{RGB}{255, 185, 196}
\definecolor{LightRed}{RGB}{255, 211, 212}

% Luxury Gradient Color Scheme
\definecolor{PhilippineBronze}{RGB}{112, 54, 14}
\definecolor{GoldenBrown}{RGB}{147, 93, 35}
\definecolor{UniversityOfCaliforniaGold}{RGB}{182, 131, 55}
\definecolor{Sunray}{RGB}{217, 170, 76}
\definecolor{OrangeYellowCrayola}{RGB}{252, 208, 96}

% Breaking Apart Color Scheme
\definecolor{DeepTuscanRed}{RGB}{103, 74, 82}
\definecolor{FuzzyWuzzy}{RGB}{199, 114, 113}
\definecolor{MiddleYellowRed}{RGB}{232, 187, 109}
\definecolor{Independence}{RGB}{83, 70, 103}
\definecolor{CeladonGreen}{RGB}{47, 129, 133}

% Lilac
\definecolor{Lilac1}{HTML}{97cded} % blue
\definecolor{Lilac2}{HTML}{02a0da}
\definecolor{Lilac3}{HTML}{b8afc9} % purple
\definecolor{Lilac4}{HTML}{eae1ef}
\definecolor{Lilac5}{HTML}{eadbd7} % light coffee

% Monochrome Coffee
\definecolor{MonoCoffee1}{HTML}{281E15} % dark
\definecolor{MonoCoffee2}{HTML}{47423E}
\definecolor{MonoCoffee3}{HTML}{928477}
\definecolor{MonoCoffee4}{HTML}{B7B2AC}
\definecolor{MonoCoffee5}{HTML}{A49FA3} % light

% Monochrome Amethyst
\definecolor{MonoAmethyst1}{HTML}{22222E} % dark
\definecolor{MonoAmethyst2}{HTML}{393A5A}
\definecolor{MonoAmethyst3}{HTML}{706F8E}
\definecolor{MonoAmethyst4}{HTML}{ADA9BA}
\definecolor{MonoAmethyst5}{HTML}{E9E9E9} % light

% Warm Brown
\definecolor{WarmBrown1}{HTML}{3C0906} % dark
\definecolor{WarmBrown2}{HTML}{84352D}
\definecolor{WarmBrown3}{HTML}{BC5F41}
\definecolor{WarmBrown4}{HTML}{E4E4E6}
\definecolor{WarmBrown5}{HTML}{E4C08A} % light

% FMAS course
\definecolor{GameTheory}{HTML}{AA2B1D}
\definecolor{Voting}{HTML}{FE9801}
\definecolor{Matching}{HTML}{F9B384}
\definecolor{Auctions}{HTML}{583D72}
\definecolor{ABMs}{HTML}{374045}
\definecolor{WisdomOfCrowds}{HTML}{7189BF}

% environments
\definecolor{colorTheorem}{HTML}{DA2D2D}
\definecolor{colorProposition}{HTML}{FF5733}
\definecolor{colorProof}{HTML}{621055}
\definecolor{colorDefinition}{HTML}{FF9F45}
\definecolor{colorAxiom}{HTML}{506D84}
\definecolor{colorProcedure}{HTML}{753422}
\definecolor{colorProblem}{HTML}{374045}
\definecolor{colorObservation}{HTML}{EA9ABB}
\definecolor{colorABM}{HTML}{89B5AF}
\definecolor{colorExample}{HTML}{97C4B8}
\definecolor{colorConjecture}{RGB}{217, 170, 76}


\newcommand{\EXP}{\mathbb{E}}
\newcommand{\maj}{{\textit{maj}}}
\renewcommand{\epsilon}{\varepsilon}




\begin{document}
    \raggedright
    %% APPORTIONMENT
    %% model
    % \begin{table}
    %     \begin{tabular}{rl}
    %         states                                  & \(N = \{1, \dots, n\}\)   \\
    %         population of state \(i\)               & \(p_i\)                   \\ 
    %         total population                        & \(p = p_1 + \dots + p_n\) \\
    %         number of seats to be allocated         & \(k\)                     \\
    %         seats allocated to state \(i\)          & \(k_i\)                   \\ 
    %         divisor                                 & \(d\)                      \\
    %         quota of state \(i\), for divisor \(d\) & \(q_i = \nicefrac{p_i}{d}\) \\
    %         standard (true) quota of state \(i\)    & \(\hat{q}_i = \nicefrac{p_i}{p}\cdot k\)\\ 
    %         upper quota of state \(i\)              & \(\lceil \hat{q}_i \rceil\), i.e., \(\hat{q}_i\) rounded up to the nearest integer \\
    %         lower quota of state \(i\)              & \(\lfloor \hat{q}_i \rfloor\), i.e., \(\hat{q}_i\) rounded down to the nearest integer
    %     \end{tabular}
    % \end{table}


    % There are \(k\) \emph{seats} to be divided among a set \(N = \{1, \dots , n\}\) 
    % of \emph{states}.
    % State \(i\) has \emph{population} \(p_i\), and the \emph{total population} is:
    % \[
    %     p = p_1 + \dots + p_n.
    % \]
    % An \emph{apportionment method} assigns to state \(i\) a number \(k_i\) of seats, such that:
    % \[
    %     k = k_1 + \dots + k_n.
    % \]
    % Given a \emph{divisor} \(d\), which stands for the desired number of people per 
    % representative, the \emph{quota} of state \(i\) is:
    % \[
    %     q_i = \nicefrac{p_i}{d}.
    % \]
    % The \emph{standard quota} of state \(i\) is:
    % \[
    %     \hat{q}_i = \nicefrac{p_i}{p} \cdot k,
    % \]
    % corresponding to the \emph{standard divisor} \(\hat{d} = \nicefrac{p}{k}\).
    % \newline

    % The \emph{upper quota} of state \(i\) is \(\lceil\hat{q}_i\rceil\),
    % and the \emph{lower quota} is \(\lfloor\hat{q}_i\rfloor\).


    % Choose a \emph{rounding function} \(f \colon \mathbb{R} \rightarrow \mathbb{Z}\).
    % It can be any function, as long as: 
    % \begin{itemize}
    %     \item[(\(i\))] \(f(x) = x\), if \(x \in \mathbb{Z}\),
    %     \item[(\(ii\))] if \(x \leq y\), then \(f(x) \leq f(y)\).
    % \end{itemize}

    % Given a rounding function \(f\), a \emph{divisor method} looks for a divisor \(d\) 
    % such that the quotas, when rounded according to \(f\), add up to the total number of seats :
    % \[
    %     f{\left(\frac{p_1}{d}\right)} + \dots + f{\left(\frac{p_n}{d}\right)} = k.
    % \]
    % State \(i\) then gets \(k_i = f{\left(\nicefrac{p_i}{d}\right)}\) seats.


    % Standard apportionment methods are given by the following rounding functions:
    % \begin{center}
    %     \begin{tabular}{rl}
    %         \textbf{Jefferson}          & \(f(x) = \lfloor x\rfloor\)\\[0.6cm]

    %         \textbf{Adams}              & \(f(x) = \lceil x\rceil\)\\[0.6cm]

    %         \textbf{Webster-Willcox}    & \(f(x) = \left[ x\right] = \begin{cases}
    %             \lfloor x\rfloor,\quad\text{if}\,\, x < \frac{\left(\lfloor x\rfloor + \lceil x\rceil\right)}{2} \\ 
    %             \lceil x \rceil,\quad\text{otherwise}.
    %         \end{cases}\)\\[0.6cm]

    %         \textbf{Huntington-Hill}    & \(f(x) = \begin{cases}
    %             \lfloor x\rfloor,\quad\text{if}\,\, x < \sqrt{\lfloor x\rfloor \cdot \lceil x\rceil} \\ 
    %             \lfloor x \rfloor,\quad\text{otherwise}.
    %         \end{cases}\)\\[0.6cm]

    %         \textbf{Dean}    & \(f(x) = \begin{cases}
    %             \lfloor x\rfloor,\quad\text{if}\,\, x < \frac{2 \cdot \lfloor x\rfloor \cdot \lceil x\rceil}{\lfloor x\rfloor + \lceil x \rceil} \\ 
    %             \lfloor x \rfloor,\quad\text{otherwise}.
    %         \end{cases}\)
    %     \end{tabular}        
    % \end{center}



    % Hamilton's method starts by assigning each state \(i\) its \emph{lower quota} \(\lfloor \hat{q}_i \rfloor\).
    % At this point the total number of seats assigned is:
    % \[
    %     \hat{k} = \lfloor \hat{q}_1 \rfloor + \dots + \lfloor \hat{q}_n \rfloor.
    % \]
    % The problem, now, is that we might be left with some unallocated seats: it's possible that \(\hat{k} < k\).
    % \newline

    % In the second step, states are ordered by their \emph{residue}:
    % \[
    %     r_i = \hat{q}_i - \lfloor \hat{q}_i \rfloor,
    % \]
    % i.e., the fractional part. The remaining seats go to the states with the highest residues.
    % \newline

    % Note that if a state \(j\) gets an additional seat in the second step, its assigned number of seats becomes:
    % \begin{align*}
    %     k_j & = \lfloor \hat{q_j}\rfloor + 1& \\
    %                & = \lceil \hat{q_j} \rceil.
    % \end{align*}
    % So with Hamilton's method every state gets either its lower quota, or its upper quota.



    % There are \(k\) \emph{seats} to be divided among a set \(N = \{1, \dots , n\}\) 
    % of \st{\emph{states}} \emph{parties}.
    % Party \(i\) has \st{\emph{population}} \emph{vote} \(v_i\), and the \emph{total number of votes} is:
    % \[
    %     v = v_1 + \dots + v_n.
    % \]
    % An \emph{apportionment method} assigns to party \(i\) a number \(k_i\) of seats, such that:
    % \[
    %     k = k_1 + \dots + k_n.
    % \]


    % Seat allocation is done sequentially. The assignment at round \(t\) is written as:
    % \[
    %     \bm{k}^t = (k_1^t, \dots , k_n^t).
    % \]
    % Start with \(\bm{k}^0 = (0, \dots, 0)\). At every round \(t \geq 1\), 
    % calculate the \emph{quotient} of party \(i\) as:
    % \[
    %     q_i^t = \frac{v_i}{k_i^{t-1} + 1}.
    % \]
    % At the end of the round, the party \(j\) with the highest quotient gets an additional seat:
    % \[
    %     k_j^t = k_j^{t-1} + 1.
    % \]
    % Stop after all seats have been allocated, \(k\) rounds in total.


    Works in a similar way to D'Hondt, except that quotients are updates as:
    \[
        q_i^t = \frac{v_i}{2k_i^{t-1} + 1}.
    \]

    


    % If a state has population \(60\)K and two representatives in the House,
    % that means one representative per \(30\)K people.
    % \vspace{0.7em}

    % If the state gets only one representative, that means \(0.5\) representatives per \(30\)K people.\\ 
    % \vspace{0.7em}

    % With three representatives, that means \(1.5\) representatives per \(30\)K people.\\
    % \vspace{0.7em}

    % In general, if the population is \(p_i\), then \(k_i\) representatives
    % means \(x_i = 30{,}000 \cdot \nicefrac{k_i}{p_i}\) representatives per \(30\)K people.\\
    % \vspace{0.7em}

    % The US Constitution requires at least one representative per state and \(x_i \leq 1\),
    % which is equivalent to:
    % \[
    %     1 \leq k_i \leq \frac{p_i}{30{,}000}.
    % \]




    % The preferred approach at the time was to start with a \emph{divisor} \(d\),
    % which stands for the desired number of people per representative.
    % \newline

    % So if a state has population \(p_i\), the number \(k_i\) 
    % of representatives it gets should be:
    % \[
    %     k_i\approx\frac{p_i}{d}.
    % \]
    % The US Constitution requires only that \(d \geq 30{,}000\),
    % and \(k_i \geq 1\).



    % Choose a divisor \(d\).
    % \newline
    
    % If state \(i\) has population \(p_i\),
    % then \(i\)'s \emph{quota} is:
    % \[
    %     q_i = \frac{p_i}{d}.
    % \]



    % If a state \(i\) with population \(p_i\) gets \(k_i\) representatives,
    % then the \emph{representation ratio} of \(i\) is:
    % \[
    %     \frac{p_i}{k_i},
    % \]
    % i.e., the number of people per representative.



    % If the total population (across all states) is \(p\), and 
    % we want to assign \(k\) seats, the \emph{standard divisor} is:
    % \[
    %     \hat{d} = \frac{p}{k}.
    % \]
    % If a state \(i\) has population \(p_i\), \(i\)'s \emph{standard quota} is:
    % \[
    %     \hat{q}_i = \frac{p_i}{\hat{d}}.
    % \]




    % \(r_i = \hat{q}_i - \lfloor\hat{q}_i\rfloor\)

    

    % \(\left\lfloor\frac{p_1}{d}\right\rfloor + \dots + \left\lfloor\frac{p_n}{d}\right\rfloor = k\)
    % \(
    %     \left[\frac{p_1}{d}\right] + \dots + \left[\frac{p_n}{d}\right] = k
    % \)
    % 
    % \(\left[\nicefrac{p_i}{d}\right]\)
    % 
    % \(s_1 = \left(\nicefrac{1}{10}, \nicefrac{9}{10}\right)\)
    % \(s_2 = \left(\nicefrac{1}{5}, \nicefrac{4}{5}\right)\)
    % \(\bm{s} = \left(s_1, s_2\right)\)
    % 
    % \(
    %     f(x) = 
    %     \begin{cases}
    %         \lfloor x \rfloor, ~\text{if}~x < \sqrt{\lfloor x \rfloor \cdot \lceil x \rceil}, \\
    %         \lceil x \rceil, ~\text{if}~x \geq \sqrt{\lfloor x \rfloor \cdot \lceil x \rceil}.
    %     \end{cases}
    % \)

    % \(f{\left(\frac{p_1}{d}\right)} + \dots +f{\left(\frac{p_n}{d}\right)} = k.\)
    % \(f{\left(\nicefrac{p_i}{d} \right)}\)


    % More generally, we can think of \(f\) as a \emph{rounding function} \(f \colon \mathbb{R} \rightarrow \mathbb{Z}\).
    % \newline

    % It can be anything, as long as it satisfies these properties:
    % \begin{itemize}
    %     \item[($i$)] \(f(x) = x\), if \(x\) is an integer,
    %     \item[($ii$)] if \(x \geq y\), then \(f(x) \geq f(y)\).
    % \end{itemize}

    % A divisor method is the Huntington-Hill method if and only if
    % for all states \(i, j \in N\) such that \(\nicefrac{p_i}{k_i} \geq \nicefrac{p_j}{k_j}\),
    % it holds that:
    % \[
    %     \frac{\nicefrac{p_i}{k_i}}{\nicefrac{p_j}{k_j}} < \frac{\nicefrac{p_j}{(k_j-1)}}{\nicefrac{p_i}{(k_i+1)}}.
    % \]
\end{document}