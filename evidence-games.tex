%% magick convert -density 1200 test.pdf test.png
\documentclass[
    preview, 
    varwidth=8.5cm, 
    border={0pt 1pt 1pt 1pt}
    ]{standalone} % border options are {left bottom right top}


% \usepackage[T1]{fontenc}
% \usepackage[sfdefault]{AlegreyaSans}
% \renewcommand*\oldstylenums[1]{{\AlegreyaSansOsF #1}}

% \usepackage[sfdefault]{FiraSans} %% option 'sfdefault' activates Fira Sans as the default text font
% \usepackage[T1]{fontenc}
% \renewcommand*\oldstylenums[1]{{\firaoldstyle #1}}

\usepackage[T1]{fontenc}
\usepackage{Alegreya} %% Option 'black' gives heavier bold face 
\renewcommand*\oldstylenums[1]{{\AlegreyaOsF #1}}


% \usepackage[default]{lato}
% \usepackage[T1]{fontenc}

% \usepackage[sfdefault]{cabin}
% \usepackage[T1]{fontenc}

% \usepackage[sfdefault]{roboto}  %% Option 'sfdefault' only if the base font of the document is to be sans serif
% \usepackage[T1]{fontenc}

% \usepackage[sfdefault]{noto}
% \usepackage[T1]{fontenc}

% \usepackage[T1]{fontenc}
% \usepackage[sfdefault]{josefin}

\usepackage{amsthm}
\usepackage{amssymb}
\usepackage{amsmath}
\usepackage{bm}
\usepackage{nicefrac}
\usepackage{graphicx}
\usepackage{tikz}
\usepackage{booktabs}
\usepackage{soul}

\usetikzlibrary{automata, positioning}

%% Colors
% grey
\definecolor{DarkCharcoal}{RGB}{30, 30, 30} % theorem
\definecolor{Davys}{RGB}{60, 60, 60} % proof
\definecolor{SonicSilver}{RGB}{90, 90, 90} % proposition
\definecolor{Gray}{RGB}{120, 120, 120} % lemma
\definecolor{Spanish}{RGB}{150, 150, 150} % corollary
\definecolor{Argent}{RGB}{180, 180, 180} % example

% blues
\definecolor{Azure}{RGB}{0, 128, 255}
\definecolor{Egyptian}{RGB}{16, 52, 166}
\definecolor{Navy}{RGB}{17, 30, 108}
\definecolor{CeruleanFrost}{RGB}{104, 145, 195}
\definecolor{LightSteelBlue}{RGB}{170, 197, 226}

% reds
\definecolor{Renate}{RGB}{227, 25, 113}
\definecolor{Burgundy}{RGB}{141, 2, 31}
\definecolor{Salmon}{RGB}{250, 128, 114}

% yellows
\definecolor{Honey}{RGB}{235, 150, 5}

% greens
\definecolor{PastelGreen}{HTML}{6ECB63}
\definecolor{FernGreen}{HTML}{116530}
\definecolor{CadmiumGreen}{HTML}{046C41}


%% Color Schemes
% Pastel Tones Color Scheme
\definecolor{Manatee}{RGB}{154, 145, 172} % darkest
\definecolor{Lilac}{RGB}{202, 167, 189}
\definecolor{CherryBlossomPink}{RGB}{255, 185, 196}
\definecolor{LightRed}{RGB}{255, 211, 212}

% Luxury Gradient Color Scheme
\definecolor{PhilippineBronze}{RGB}{112, 54, 14}
\definecolor{GoldenBrown}{RGB}{147, 93, 35}
\definecolor{UniversityOfCaliforniaGold}{RGB}{182, 131, 55}
\definecolor{Sunray}{RGB}{217, 170, 76}
\definecolor{OrangeYellowCrayola}{RGB}{252, 208, 96}

% Breaking Apart Color Scheme
\definecolor{DeepTuscanRed}{RGB}{103, 74, 82}
\definecolor{FuzzyWuzzy}{RGB}{199, 114, 113}
\definecolor{MiddleYellowRed}{RGB}{232, 187, 109}
\definecolor{Independence}{RGB}{83, 70, 103}
\definecolor{CeladonGreen}{RGB}{47, 129, 133}

% Lilac
\definecolor{Lilac1}{HTML}{97cded} % blue
\definecolor{Lilac2}{HTML}{02a0da}
\definecolor{Lilac3}{HTML}{b8afc9} % purple
\definecolor{Lilac4}{HTML}{eae1ef}
\definecolor{Lilac5}{HTML}{eadbd7} % light coffee

% Monochrome Coffee
\definecolor{MonoCoffee1}{HTML}{281E15} % dark
\definecolor{MonoCoffee2}{HTML}{47423E}
\definecolor{MonoCoffee3}{HTML}{928477}
\definecolor{MonoCoffee4}{HTML}{B7B2AC}
\definecolor{MonoCoffee5}{HTML}{A49FA3} % light

% Monochrome Amethyst
\definecolor{MonoAmethyst1}{HTML}{22222E} % dark
\definecolor{MonoAmethyst2}{HTML}{393A5A}
\definecolor{MonoAmethyst3}{HTML}{706F8E}
\definecolor{MonoAmethyst4}{HTML}{ADA9BA}
\definecolor{MonoAmethyst5}{HTML}{E9E9E9} % light

% Warm Brown
\definecolor{WarmBrown1}{HTML}{3C0906} % dark
\definecolor{WarmBrown2}{HTML}{84352D}
\definecolor{WarmBrown3}{HTML}{BC5F41}
\definecolor{WarmBrown4}{HTML}{E4E4E6}
\definecolor{WarmBrown5}{HTML}{E4C08A} % light

% FMAS course
\definecolor{GameTheory}{HTML}{AA2B1D}
\definecolor{Voting}{HTML}{FE9801}
\definecolor{Matching}{HTML}{F9B384}
\definecolor{Auctions}{HTML}{583D72}
\definecolor{ABMs}{HTML}{374045}
\definecolor{WisdomOfCrowds}{HTML}{7189BF}

% environments
\definecolor{colorTheorem}{HTML}{DA2D2D}
\definecolor{colorProposition}{HTML}{FF5733}
\definecolor{colorProof}{HTML}{621055}
\definecolor{colorDefinition}{HTML}{FF9F45}
\definecolor{colorAxiom}{HTML}{506D84}
\definecolor{colorProcedure}{HTML}{753422}
\definecolor{colorProblem}{HTML}{374045}
\definecolor{colorObservation}{HTML}{EA9ABB}
\definecolor{colorABM}{HTML}{89B5AF}
\definecolor{colorExample}{HTML}{97C4B8}
\definecolor{colorConjecture}{RGB}{217, 170, 76}


\newcommand{\EXP}{\mathbb{E}}
\newcommand{\maj}{{\textit{maj}}}
\renewcommand{\epsilon}{\varepsilon}


\begin{document}\raggedright
    % \emph{Nature} determines the state:
    % \emph{High} with probability \(p\) and \emph{Low} with probability \(1 - p\).

    % \vspace{0.2cm}
    % Player \(1\) can get evidence of the state:
    % with probability \(q_h\) if state is High,
    % and \(q_\ell\) if state is Low.
    % In the remaining cases Player \(1\) gets no evidence.

    % \vspace{0.2cm}
    % Player \(1\) sees only the evidence, if any, 
    % but not the state itself (dashed line).

    % \vspace{0.2cm}
    % If in possession of evidence, Player \(1\) chooses whether to reveal it (R or \(\lnot\)R)
    % to Player \(2\). If Player \(1\) hasn't gotten any evidence,
    % they don't reveal anything (\(\lnot\)R).

    % \vspace{0.2cm}
    % Player \(2\) observes only Player \(1\)'s action (\(\textsf{X} \in \{\textsf{R}, \lnot \textsf{R}\}\)),
    % but not the state or the evidence (dashed lines).

    % \vspace{0.2cm}
    % Based on Player \(1\)'s action, Player \(2\) updates 
    % their posterior belief about the state:
    % \[
    %     {\Pr}\Big[ \text{High} \mid \textsf{X} \Big].
    % \]
    % Player \(1\)'s utility is Player \(2\)'s posterior about the state being High.\(^*\)



    % Consider the strategy profile in which:
    % \begin{itemize}
    %     \item Player \(1\) reveals the evidence when getting it, and (obviously) withholds it otherwise;
    %     \item Player \(2\) updates their posterior---expecting that Player \(1\) reveals the evidence when they have it, and does not reveal it otherwise.
    % \end{itemize}
    % Player \(2\)'s posteriors, given they expect Player \(1\) to follow this strategy, are:
    % \begin{flalign*}
    %     \qquad
    %     {\Pr}\Big[ \text{High} \mid \textsf{R} \Big] &=
    %     \frac{{\Pr}\big[ \textsf{R} \mid \textsf{High} \big] \cdot {\Pr}\big[ \textsf{High}\big]}{{\Pr}\big[ \textsf{R}\big]} 
    %     = \frac{q_h \cdot p}{q_h \cdot p + (1 - q_h) (1 - p)}, &\\
    %     {\Pr}\Big[ \text{High} \mid \lnot \textsf{R} \Big] 
    %     & =\frac{{\Pr}\big[ \lnot\textsf{R} \mid \textsf{High} \big] \cdot {\Pr}\big[ \textsf{High}\big]}{{\Pr}\big[ \lnot\textsf{R}\big]}
    %     =\frac{(1 - q_h) p}{(1 - q_h) p + (1-q_\ell) (1 - p)}.&
    % \end{flalign*}
    % This is an equilibrium as long as Player \(1\) does not want to deviate,
    % which happens as long as \({\Pr}\big[ \text{High} \mid \textsf{R} \big] \geq {\Pr}\big[ \text{High} \mid \lnot \textsf{R} \big]\). 
    
    % \vspace{0.2cm}
    % Equivalent to:
    % \begin{flalign*}
    %     \qquad
    %     \frac{q_h \cdot p}{q_h \cdot p + (1 - q_h) (1 - p)} \geq \frac{(1 - q_h) p}{(1 - q_h) p + (1-q_\ell) (1 - p)} & \text{ iff } q_h \geq q_\ell.&
    % \end{flalign*}





    % Consider the alternative strategy profile in which:
    % \begin{itemize}
    %     \item Player \(1\) never reveals anything, even when they have evidence;
    %     \item Player \(2\) updates their posterior---expecting that Player \(1\) never reveals.
    % \end{itemize}
    % Player \(2\)'s posteriors, given they expect Player \(1\) to follow this strategy, are:
    % \begin{flalign*}
    %     \qquad
    %     {\Pr}\Big[ \text{High} \mid \textsf{R} \Big] &=
    %     \frac{{\Pr}\big[ \textsf{R} \mid \textsf{High} \big] \cdot {\Pr}\big[ \textsf{High}\big]}{{\Pr}\big[ \textsf{R}\big]} 
    %     = \frac{0 \cdot p}{0 \cdot p + (1 - q_h) (1 - p)} = 0, &\\
    %     {\Pr}\Big[ \text{High} \mid \lnot \textsf{R} \Big] 
    %     & =\frac{{\Pr}\big[ \lnot\textsf{R} \mid \textsf{High} \big] \cdot {\Pr}\big[ \textsf{High}\big]}{{\Pr}\big[ \lnot\textsf{R}\big]}
    %     =\frac{1 \cdot p}{1} = p.&
    % \end{flalign*}
    % Player \(1\) does not want to switch to revealing, so this is also an equilibrium.




    % Nature determines the state (High or Low) with probability \(p\)
    % and generates evidence with probabilities \(q_h\) and \(q_\ell\), respectively.

    % \vspace{0.2cm}
    % Player \(1\) does not have immediate access to the evidence, but has to search for it.

    % \vspace{0.2cm}
    % Player \(1\) can search for evidence, 
    % using two levels of effort: Max, which involves cost \(c\),
    % and Min, which involves cost \(0\).

    % \vspace{0.2cm}
    % If it exists, evidence is obtained (O) with probabilities \(f_{max}\) or \(f_{min}\), 
    % depending on the effort expended (\(f_{max} > f_{min}\)). In the remaining cases,
    % no evidence is obtained (\(\lnot\)O).

    % \vspace{0.2cm}
    % Player \(1\) chooses whether to reveal evidence (R or \(\lnot\)R), if obtained.
    % If no evidence is obtained, Player \(1\) does not reveal anything (\(\lnot\)R).


    % \vspace{0.2cm}
    % Player \(2\) updates their posterior belief about the state depending on whether 
    % Player \(1\) reveals evidence or not.

    % \vspace{0.2cm}
    % Player \(1\)'s utility is Player \(2\)'s posterior about the state, together with the search cost.





    % Consider the strategy profile in which:
    % \begin{itemize}
    %     \item Player \(1\) expends Max effort and reveals the evidence when getting it;
    %     \item Player \(2\) updates their posterior as usual.
    % \end{itemize}
    

    % \vspace{0.2cm}
    % This is an equilibrium as long as:
    % \begin{flalign*}
    %     \quad
    %     c  \leq \big(f_{max} - f_{min}\big) \cdot {\Pr}\big[ \textsf{Ev} \big]
    %     \cdot \bigg({\Pr}\big[ \text{High} \mid \textsf{R} \big] - {\Pr}\big[ \text{High} \mid \lnot\textsf{R} \big]\bigg).&&
    % \end{flalign*}





    % Nature determines the state (High or Low) with probability \(p\).
    % Player \(1\) does not know the state.

    % \vspace{0.2cm}
    % Player \(1\) can choose a test \((q_h, q_\ell)\), with \(q_h > q_\ell\).

    % \vspace{0.2cm}
    % Evidence is generated according to the chosen test, and the state of the world.

    % \vspace{0.2cm}
    % If evidence is generated, Player \(1\) automatically observes it and chooses whether to 
    % reveal it to Player~\(2\).

    % \vspace{0.2cm}
    % Player \(2\) updates their posterior belief about the state depending on whether 
    % Player \(1\) reveals evidence or not, and their beliefs about the test and state.

    % \vspace{0.2cm}
    % Player \(1\)'s utility is Player \(2\)'s posterior about the state.




    Consider the strategy profile in which:
    \begin{itemize}
        \item Player \(1\) chooses a test that maximizes \(q_h\) and \(q_\ell\), 
                and reveals evidence when generated.
        \item Player \(2\) updates as usual.
    \end{itemize}
    

    \vspace{0.2cm}
    This is an equilibrium.

\end{document}