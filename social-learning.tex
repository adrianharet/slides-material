%% magick convert -density 1200 test.pdf test.png
\documentclass[preview, border={0pt 5pt 0pt 1pt}, varwidth=13cm]{standalone} % border options are {left bottom right top}

% \usepackage[T1]{fontenc}
% \usepackage[sfdefault]{AlegreyaSans}
% \renewcommand*\oldstylenums[1]{{\AlegreyaSansOsF #1}}

% \usepackage[sfdefault]{FiraSans} %% option 'sfdefault' activates Fira Sans as the default text font
% \usepackage[T1]{fontenc}
% \renewcommand*\oldstylenums[1]{{\firaoldstyle #1}}

\usepackage[T1]{fontenc}
\usepackage{Alegreya} %% Option 'black' gives heavier bold face 
\renewcommand*\oldstylenums[1]{{\AlegreyaOsF #1}}


% \usepackage[default]{lato}
% \usepackage[T1]{fontenc}

% \usepackage[sfdefault]{cabin}
% \usepackage[T1]{fontenc}

% \usepackage[sfdefault]{roboto}  %% Option 'sfdefault' only if the base font of the document is to be sans serif
% \usepackage[T1]{fontenc}

% \usepackage[sfdefault]{noto}
% \usepackage[T1]{fontenc}

% \usepackage[T1]{fontenc}
% \usepackage[sfdefault]{josefin}

\usepackage{amsthm}
\usepackage{amssymb}
\usepackage{amsmath}
\usepackage{bm}
\usepackage{nicefrac}
\usepackage{graphicx}
\usepackage{tikz}
\usepackage{booktabs}
\usepackage{soul}

\usetikzlibrary{automata, positioning}

%% Colors
% grey
\definecolor{DarkCharcoal}{RGB}{30, 30, 30} % theorem
\definecolor{Davys}{RGB}{60, 60, 60} % proof
\definecolor{SonicSilver}{RGB}{90, 90, 90} % proposition
\definecolor{Gray}{RGB}{120, 120, 120} % lemma
\definecolor{Spanish}{RGB}{150, 150, 150} % corollary
\definecolor{Argent}{RGB}{180, 180, 180} % example

% blues
\definecolor{Azure}{RGB}{0, 128, 255}
\definecolor{Egyptian}{RGB}{16, 52, 166}
\definecolor{Navy}{RGB}{17, 30, 108}
\definecolor{CeruleanFrost}{RGB}{104, 145, 195}
\definecolor{LightSteelBlue}{RGB}{170, 197, 226}

% reds
\definecolor{Renate}{RGB}{227, 25, 113}
\definecolor{Burgundy}{RGB}{141, 2, 31}
\definecolor{Salmon}{RGB}{250, 128, 114}

% yellows
\definecolor{Honey}{RGB}{235, 150, 5}

% greens
\definecolor{PastelGreen}{HTML}{6ECB63}
\definecolor{FernGreen}{HTML}{116530}
\definecolor{CadmiumGreen}{HTML}{046C41}


%% Color Schemes
% Pastel Tones Color Scheme
\definecolor{Manatee}{RGB}{154, 145, 172} % darkest
\definecolor{Lilac}{RGB}{202, 167, 189}
\definecolor{CherryBlossomPink}{RGB}{255, 185, 196}
\definecolor{LightRed}{RGB}{255, 211, 212}

% Luxury Gradient Color Scheme
\definecolor{PhilippineBronze}{RGB}{112, 54, 14}
\definecolor{GoldenBrown}{RGB}{147, 93, 35}
\definecolor{UniversityOfCaliforniaGold}{RGB}{182, 131, 55}
\definecolor{Sunray}{RGB}{217, 170, 76}
\definecolor{OrangeYellowCrayola}{RGB}{252, 208, 96}

% Breaking Apart Color Scheme
\definecolor{DeepTuscanRed}{RGB}{103, 74, 82}
\definecolor{FuzzyWuzzy}{RGB}{199, 114, 113}
\definecolor{MiddleYellowRed}{RGB}{232, 187, 109}
\definecolor{Independence}{RGB}{83, 70, 103}
\definecolor{CeladonGreen}{RGB}{47, 129, 133}

% Lilac
\definecolor{Lilac1}{HTML}{97cded} % blue
\definecolor{Lilac2}{HTML}{02a0da}
\definecolor{Lilac3}{HTML}{b8afc9} % purple
\definecolor{Lilac4}{HTML}{eae1ef}
\definecolor{Lilac5}{HTML}{eadbd7} % light coffee

% Monochrome Coffee
\definecolor{MonoCoffee1}{HTML}{281E15} % dark
\definecolor{MonoCoffee2}{HTML}{47423E}
\definecolor{MonoCoffee3}{HTML}{928477}
\definecolor{MonoCoffee4}{HTML}{B7B2AC}
\definecolor{MonoCoffee5}{HTML}{A49FA3} % light

% Monochrome Amethyst
\definecolor{MonoAmethyst1}{HTML}{22222E} % dark
\definecolor{MonoAmethyst2}{HTML}{393A5A}
\definecolor{MonoAmethyst3}{HTML}{706F8E}
\definecolor{MonoAmethyst4}{HTML}{ADA9BA}
\definecolor{MonoAmethyst5}{HTML}{E9E9E9} % light

% Warm Brown
\definecolor{WarmBrown1}{HTML}{3C0906} % dark
\definecolor{WarmBrown2}{HTML}{84352D}
\definecolor{WarmBrown3}{HTML}{BC5F41}
\definecolor{WarmBrown4}{HTML}{E4E4E6}
\definecolor{WarmBrown5}{HTML}{E4C08A} % light

% FMAS course
\definecolor{GameTheory}{HTML}{AA2B1D}
\definecolor{Voting}{HTML}{FE9801}
\definecolor{Matching}{HTML}{F9B384}
\definecolor{Auctions}{HTML}{583D72}
\definecolor{ABMs}{HTML}{374045}
\definecolor{WisdomOfCrowds}{HTML}{7189BF}

% environments
\definecolor{colorTheorem}{HTML}{DA2D2D}
\definecolor{colorProposition}{HTML}{FF5733}
\definecolor{colorProof}{HTML}{621055}
\definecolor{colorDefinition}{HTML}{FF9F45}
\definecolor{colorAxiom}{HTML}{506D84}
\definecolor{colorProcedure}{HTML}{753422}
\definecolor{colorProblem}{HTML}{374045}
\definecolor{colorObservation}{HTML}{EA9ABB}
\definecolor{colorABM}{HTML}{89B5AF}
\definecolor{colorExample}{HTML}{97C4B8}
\definecolor{colorConjecture}{RGB}{217, 170, 76}


\newcommand{\EXP}{\mathbb{E}}
\newcommand{\maj}{{\textit{maj}}}
\renewcommand{\epsilon}{\varepsilon}


\begin{document}\raggedright

    %% Model
    % \begin{table}
    %     \begin{tabular}{rl}
    %         agents & \(1, 2, \dots , n\) \\
    %         [0.4em]
    %         time & \(t \in \{0, 1, 2, \dots \} \) \\
    %         [0.4em]
    %         true state & \(\mu \in (0, 1)\) \\
    %         [0.4em]
    %         belief of agent \(i\) at \(t\) & number between \(0\) and \(1\)\\
    %                                        & drawn from a distribution with mean \(\mu\) \\
    %                                        & and finite variance above a threshold \(\delta > 0\)\\
    %         [0.4em]
    %         social network & aperiodic, strongly connected directed graph with agents as vertices, \\ 
    %                        & and who-pays-attention-to-who as edges \\
    %         [0.4em]
    %         agent \(i\)'s neighborhood & agents that \(i\) pays attention to \\
    %         [0.4em]
    %         weight on edge from \(i\) to \(j\) & number that indicates how much weight \(i\) places on \(j\)'s opinion;\\ 
    %                                            & we assume \(i\) distributes a total weight of \(1\) across \(i\)'s neighborhood\\
    %         [0.4em]
    %         update rule & at time \(t+1\) every agent updates their belief \\ 
    %                     & to a weighted average over the beliefs of neighbors
    %     \end{tabular}
    % \end{table}

    % There is a set \(N = \{1, 2, \dots, n\}\) of \emph{agents}.
    % Each agent \(i\) has an \emph{opinion}, or \emph{belief}, \(x_i \in [0,1]\).
    % The opinions are meant to track a \emph{true state} \(\mu \in (0, 1)\).
    % \vspace{1em}

    % Time goes by in discrete steps \(t \in \{0, 1, 2, \dots \}\).
    % Agent \(i\)'s \emph{opinion at time} \(t\) is \(x^t_i\).
    % \vspace{1em}

    % Agents are connected by a \emph{social network} \(G = (N, E)\), which is a directed graph.
    % An edge from \(i\) to \(j\) indicates that agent \(i\) pays attention to agent \(j\).
    % Agent \(i\)'s \emph{(out-)neighborhood} \(N(i)\) is the set of agents that \(i\) pays attention to:
    % \begin{flalign*}
    %     \qquad
    %     N(i) &= \{ j \in N \mid (i, j) \in E \}. &
    % \end{flalign*}
    % Each agent \(i\) distributes a total weight of \(1\) across the agents in \(N(i)\):
    % \begin{flalign*}
    %     \qquad
    %     \sum_{j \in N(i)} w_{ij} &= 1, &
    % \end{flalign*}
    % where \(w_{ij} > 0\) is the \emph{weight} that agent \(i\) places on agent \(j\)'s opinion.
    % \vspace{1em}

    % At each new time step, agents \emph{update} their opinions to a weighted average 
    % of the opinions of agents they pay attention to:
    % \begin{flalign*}
    %     \qquad
    %     x^{t+1}_i &= \sum_{j \in N(i)} w_{ij} x^t_j. &
    % \end{flalign*}


    % Take \(N = \{1, \dots, 6\}\) to be the set of agents.
    % They are connected by the social network \(G\) on the right.
    % \vspace{1em}

    % Each agent distributes a total weight of \(1\) across the agents in their neighborhood.
    % \vspace{1em}

    % The true state is \(\mu = 0.5\).
    % Each agent starts with the initial belief shown on the right.\(^*\)
    % \vspace{1em}

    % Time starts at \(t = 0\).
    % \vspace{1em}

    % At \(t=1\), each agent updates their belief to a weighted average of the beliefs of agents they pay attention to. For instance, agent \(1\)'s belief becomes:
    % \begin{flalign*}
    %     \qquad
    %     x^1_1 &= 0.5 \cdot 0 + 0.3 \cdot 1 + 0.2 \cdot 0.2 &\\
    %           &= 0.34. &
    % \end{flalign*}
    % This keeps going for as long as we like\dots







    
    


    % The \emph{weight matrix} of network \(G\) is a matrix 
    % \(
    % W = 
    % \begin{bmatrix}
    %     w_{11} & w_{12} & \cdots & w_{1n} \\
    %     w_{21} & w_{22} & \cdots & w_{2n} \\
    %     \vdots & \vdots & \ddots & \vdots \\
    %     w_{n1} & w_{n2} & \cdots & w_{nn}
    % \end{bmatrix}
    % \), 
    % where:
    % \[
    %     w_{ij} = 
    %     \begin{cases}
    %         \text{weight that agent \(i\) places on agent \(j\)'s opinion,} & \text{if } (i, j) \in E \\
    %         0 & \text{otherwise.}
    %     \end{cases}
    % \]


    % Consider the graph on the right. The weight matrix is:
    % \begin{flalign*}
    %     \qquad
    %     W &= 
    %     \begin{bmatrix}
    %         0.5 & 0.25 & 0.25 \\
    %         1 & 0 & 0 \\
    %         1 & 0 & 0            
    %     \end{bmatrix}.&
    % \end{flalign*}
    % Take initial beliefs to be \(x_1^0 = 1\), \(x_2^0 = 0\), and \(x_3^0 = 0\).
    % \vspace{1em}

    % Beliefs at time \(t=1\) are:
    % \begin{flalign*}
    % \qquad
    %     \bm{x}^1 & = W \cdot \bm{x}^0 &\\
    %              &  = \begin{bmatrix}
    %                     0.5 & 0.25 & 0.25 \\
    %                     1 & 0 & 0 \\
    %                     1 & 0 & 0            
    %                 \end{bmatrix}
    %                 \cdot \begin{bmatrix}
    %                     1 \\
    %                     0 \\
    %                     0   
    %                 \end{bmatrix} &\\
    %             & = \begin{bmatrix}
    %                 0.5 \cdot 1 + 0.25 \cdot 0 + 0.25 \cdot 0 \\
    %                 1 \cdot 1 + 0 \cdot 0 + 0 \cdot 0 \\
    %                 1 \cdot 1 + 0 \cdot 0 + 0 \cdot 0
    %             \end{bmatrix}\\
    %             &= \begin{bmatrix}
    %                 0.5 \\
    %                 1 \\
    %                 1
    %             \end{bmatrix}.&
    % \end{flalign*}

    % Beliefs at time \(t=2\) are:
    % \begin{flalign*}
    %     \qquad
    %     \bm{x}^2 & = W \cdot \bm{x}^1 &\\
    %              & = W \cdot \left(W \cdot \bm{x}^0\right) &\\
    %              & = W^2 \cdot \bm{x}^0. &\\
    %             & = \begin{bmatrix}
    %                 0.75\\
    %                 0.5 \\
    %                 0.5 
    %             \end{bmatrix} &
    % \end{flalign*}

    % In general, beliefs at time \(t\) are:
    % \begin{flalign*}
    %     \qquad
    %     \bm{x}^t & = W^t \cdot \bm{x}^0. &
    % \end{flalign*}


    % As \(t\) goes to infinity, the beliefs converge to a limit:\(^*\)
    % \begin{flalign*}   
    %     \qquad
    %     \bm{x}^* & = \lim_{t \to \infty} W^t \cdot \bm{x}^0 & \\
    %              & = W^* \cdot \bm{x}^0. & \\
    %              & = \begin{bmatrix}
    %                  w^*_1 & w^*_2 & w_3^*\\
    %                  w^*_1 & w^*_2 & w_3^*\\
    %                  w^*_1 & w^*_2 & w_3^*
    %                 \end{bmatrix}
    %                 \cdot
    %                 \begin{bmatrix}
    %                     x^0_1 \\
    %                     x^0_2 \\
    %                     x^0_3
    %                 \end{bmatrix}.&
    % \end{flalign*}
    % \begin{flushright}
    %     \footnotesize
    %     \(^*\)Note that, since the limit belief is indenpendent of the initial beliefs, 
    %     the rows of \(W^*\) have to be equal.        
    % \end{flushright}


    % So the limit belief is:
    % \begin{flalign*}
    %     \qquad
    %     \tilde{x} & = w_1^* x^0_1 + w_2^* x^0_2 + w_3^* x^0_3 & \\
    %               & =
    %                 \begin{bmatrix}
    %                     w_1^* & w_2^* & w_3^*
    %                 \end{bmatrix}  
    %                 \cdot 
    %                 \begin{bmatrix}
    %                     x^0_1 \\
    %                     x^0_2 \\
    %                     x^0_3
    %                 \end{bmatrix} & \\
    %               & = \bm{w} \cdot \bm{x}^0.  
    % \end{flalign*}
    % Note, again, that this holds for any initial beliefs.


    % We got that the consensus belief is:
    % \begin{flalign*}
    %     \qquad
    %     \tilde{x} &= \bm{w} \cdot \bm{x}^0.&
    % \end{flalign*}
    % But note that we would get the same consensus belief
    % even if we started at \(\bm{x}^1\)!
    % \vspace{1em}
    
    % So:
    % \begin{flalign*}
    %     \qquad
    %     \bm{w} \cdot \bm{x}^0 & = \bm{w} \cdot \bm{x}^1 & \\
    %                           & = \bm{w} \cdot \left(W \cdot \bm{x}^0 \right) & \\
    %                           & = \left(\bm{w} \cdot W\right) \cdot \bm{x}^0. &
    % \end{flalign*}
    % Simplifying and rearranging gives us:
    % \begin{flalign*}
    %     \qquad
    %     \bm{w} \cdot W & = \bm{w}. &
    % \end{flalign*}
    % This means that \(\bm{w}\) is a \emph{left eigenvector} of \(W\) with eigenvalue \(1\).




    % Consider the graph on the right. The weight matrix is:
    % \begin{flalign*}
    %     \qquad
    %     W &= 
    %     \begin{bmatrix}
    %         0.5 & 0.25 & 0.25 \\
    %         1 & 0 & 0 \\
    %         1 & 0 & 0            
    %     \end{bmatrix}.&
    % \end{flalign*}
    % The eigenvector centralities of the nodes are 
    % \(c_1 = \nicefrac{2}{3}\), \(c_2 = \nicefrac{1}{6}\), and \(c_3 = \nicefrac{1}{6}\).
    % \vspace{1em}

    % Take initial beliefs to be \(x_1^0 = 1\), \(x_2^0 = 0\), and \(x_3^0 = 0\).
    % \vspace{1em}

    % If we compute the limits and do the maths, we get the consensus belief as:
    % \begin{flalign*}
    %     \qquad
    %     \tilde{x} & = \frac{2}{3}.&
    % \end{flalign*}
    % Note that we get the same from the eigenvector centralities and initial beliefs:
    % \begin{flalign*}
    %     \qquad
    %     c_1 \cdot x_1^0 + c_2 \cdot x_2^0 + c_3 \cdot x_3^0 & = \frac{2}{3} \cdot 1 + \frac{1}{6} \cdot 0 + \frac{1}{6} \cdot 0&\\
    %     & = \frac{2}{3}.&
    % \end{flalign*}
    % \vspace{1em}



    Assume a sequence \(G_1\), \(G_2\), \dots, of strongly connected and aperiodic networks of increasing size, 
    and initial beliefs drawn from a distribution with mean \(\mu\) (the true state) and 
    finite variance above a threshold \(\delta > 0\).
    \vspace{1em}

    The sequence of networks is \emph{wise} if and only if the eigenvector centrality of every agent 
    approaches \(0\) asymptotically, as \(n\) goes to infinity.



    
    %% Model for cascades
    % \begin{table}
    %     \begin{tabular}{rl}
    %         agents & \(N = \{1, \dots , n\}\) \\
    %         alternatives & \(A = \{a,b\}\) \\
    %         true alternative & \(\theta \in A\), we usually assume \(\theta = a\) \\
    %         voter \(i\)'s signal & \(s_i \in A\)\\
    %         probability of a correct signal \(i\)'s & \(\Pr[s_i = \theta ] = p\), with \(p > \nicefrac{1}{2}\) \\
    %         agents' prior probabilities & \(\Pr\left[ \theta = a \right] = \Pr\left[ \theta = b \right] = \nicefrac{1}{2}\)\\
    %         agent \(i\)'s verdict & \(v_i \in A\) \\
    %                             & agents speak out in sequence, and see previous verdicts
    %     \end{tabular}
    % \end{table}

    % \[
    %     \Pr\left[ A | B \right] = \frac{\Pr\left[ B | A \right] \cdot \Pr\left[ A \right] }{\Pr\left[ B \right]}
    % \]

    % \begin{flalign*}
    %     \Pr\left[ \theta = a \mid s_1 = a \right] & = \frac{ \Pr\left[ s_1 = a \mid \theta = a \right] \cdot \Pr\left[ \theta = a \right]}{\Pr\left[ s_1 = a \right]}& \\
    %     \Pr\left[ \theta = b \mid s_1 = a \right] & = \frac{ \Pr\left[ s_1 = a \mid \theta = b \right] \cdot \Pr\left[ \theta = b \right]}{\Pr\left[ s_1 = a \right]}&
    % \end{flalign*}

    % With \(p > \nicefrac{1}{2}\), we have that \(\Pr\left[ \theta = a \mid s_1 = a \right] > \Pr\left[ \theta = b \mid s_1 = a \right]\).


    % \begin{flalign*}
    %     \Pr\left[ \theta = b \mid s_3 = b, v_2 = v_1 = b \right] & > \Pr\left[ \theta = a \mid s_3 = b, v_2 = v_1 = b \right] &\\
    %     \Pr\left[ \theta = b \mid s_3 = a, v_2 = v_1 = b \right] & > \Pr\left[ \theta = a \mid s_3 = a, v_2 = v_1 = b \right] &
    % \end{flalign*}

    % \(\nicefrac{1}{2(n{-}1)}\)

    % The network grows by adding agents that listen to the central agent 1.\\

    % The eigenvector centralities are:

    % \[
    %     \bm{c} = \left(\frac{1}{2}, \frac{1}{2(n-1)}, \dots , \frac{1}{2(n-1)}\right)
    % \]

    % Agent 1 retains a constant share of (network) influence as \(n\) grows.\\

    % And thus decides the consensus belief.\\

    % No bueno.
\end{document}