%% magick convert -density 1200 test.pdf test.png
\documentclass[preview, border={0pt 5pt 0pt 1pt}, varwidth=10cm]{standalone} % border options are {left bottom right top}

% \usepackage[T1]{fontenc}
% \usepackage[sfdefault]{AlegreyaSans}
% \renewcommand*\oldstylenums[1]{{\AlegreyaSansOsF #1}}

% \usepackage[sfdefault]{FiraSans} %% option 'sfdefault' activates Fira Sans as the default text font
% \usepackage[T1]{fontenc}
% \renewcommand*\oldstylenums[1]{{\firaoldstyle #1}}

\usepackage[T1]{fontenc}
\usepackage{Alegreya} %% Option 'black' gives heavier bold face 
\renewcommand*\oldstylenums[1]{{\AlegreyaOsF #1}}


% \usepackage[default]{lato}
% \usepackage[T1]{fontenc}

% \usepackage[sfdefault]{cabin}
% \usepackage[T1]{fontenc}

% \usepackage[sfdefault]{roboto}  %% Option 'sfdefault' only if the base font of the document is to be sans serif
% \usepackage[T1]{fontenc}

% \usepackage[sfdefault]{noto}
% \usepackage[T1]{fontenc}

% \usepackage[T1]{fontenc}
% \usepackage[sfdefault]{josefin}

\usepackage{amsthm}
\usepackage{amssymb}
\usepackage{amsmath}
\usepackage{bm}
\usepackage{nicefrac}
\usepackage{graphicx}
\usepackage{tikz}
\usepackage{booktabs}
\usepackage{soul}

\usetikzlibrary{automata, positioning}

%% Colors
% grey
\definecolor{DarkCharcoal}{RGB}{30, 30, 30} % theorem
\definecolor{Davys}{RGB}{60, 60, 60} % proof
\definecolor{SonicSilver}{RGB}{90, 90, 90} % proposition
\definecolor{Gray}{RGB}{120, 120, 120} % lemma
\definecolor{Spanish}{RGB}{150, 150, 150} % corollary
\definecolor{Argent}{RGB}{180, 180, 180} % example

% blues
\definecolor{Azure}{RGB}{0, 128, 255}
\definecolor{Egyptian}{RGB}{16, 52, 166}
\definecolor{Navy}{RGB}{17, 30, 108}
\definecolor{CeruleanFrost}{RGB}{104, 145, 195}
\definecolor{LightSteelBlue}{RGB}{170, 197, 226}

% reds
\definecolor{Renate}{RGB}{227, 25, 113}
\definecolor{Burgundy}{RGB}{141, 2, 31}
\definecolor{Salmon}{RGB}{250, 128, 114}

% yellows
\definecolor{Honey}{RGB}{235, 150, 5}

% greens
\definecolor{PastelGreen}{HTML}{6ECB63}
\definecolor{FernGreen}{HTML}{116530}
\definecolor{CadmiumGreen}{HTML}{046C41}


%% Color Schemes
% Pastel Tones Color Scheme
\definecolor{Manatee}{RGB}{154, 145, 172} % darkest
\definecolor{Lilac}{RGB}{202, 167, 189}
\definecolor{CherryBlossomPink}{RGB}{255, 185, 196}
\definecolor{LightRed}{RGB}{255, 211, 212}

% Luxury Gradient Color Scheme
\definecolor{PhilippineBronze}{RGB}{112, 54, 14}
\definecolor{GoldenBrown}{RGB}{147, 93, 35}
\definecolor{UniversityOfCaliforniaGold}{RGB}{182, 131, 55}
\definecolor{Sunray}{RGB}{217, 170, 76}
\definecolor{OrangeYellowCrayola}{RGB}{252, 208, 96}

% Breaking Apart Color Scheme
\definecolor{DeepTuscanRed}{RGB}{103, 74, 82}
\definecolor{FuzzyWuzzy}{RGB}{199, 114, 113}
\definecolor{MiddleYellowRed}{RGB}{232, 187, 109}
\definecolor{Independence}{RGB}{83, 70, 103}
\definecolor{CeladonGreen}{RGB}{47, 129, 133}

% Lilac
\definecolor{Lilac1}{HTML}{97cded} % blue
\definecolor{Lilac2}{HTML}{02a0da}
\definecolor{Lilac3}{HTML}{b8afc9} % purple
\definecolor{Lilac4}{HTML}{eae1ef}
\definecolor{Lilac5}{HTML}{eadbd7} % light coffee

% Monochrome Coffee
\definecolor{MonoCoffee1}{HTML}{281E15} % dark
\definecolor{MonoCoffee2}{HTML}{47423E}
\definecolor{MonoCoffee3}{HTML}{928477}
\definecolor{MonoCoffee4}{HTML}{B7B2AC}
\definecolor{MonoCoffee5}{HTML}{A49FA3} % light

% Monochrome Amethyst
\definecolor{MonoAmethyst1}{HTML}{22222E} % dark
\definecolor{MonoAmethyst2}{HTML}{393A5A}
\definecolor{MonoAmethyst3}{HTML}{706F8E}
\definecolor{MonoAmethyst4}{HTML}{ADA9BA}
\definecolor{MonoAmethyst5}{HTML}{E9E9E9} % light

% Warm Brown
\definecolor{WarmBrown1}{HTML}{3C0906} % dark
\definecolor{WarmBrown2}{HTML}{84352D}
\definecolor{WarmBrown3}{HTML}{BC5F41}
\definecolor{WarmBrown4}{HTML}{E4E4E6}
\definecolor{WarmBrown5}{HTML}{E4C08A} % light

% FMAS course
\definecolor{GameTheory}{HTML}{AA2B1D}
\definecolor{Voting}{HTML}{FE9801}
\definecolor{Matching}{HTML}{F9B384}
\definecolor{Auctions}{HTML}{583D72}
\definecolor{ABMs}{HTML}{374045}
\definecolor{WisdomOfCrowds}{HTML}{7189BF}

% environments
\definecolor{colorTheorem}{HTML}{DA2D2D}
\definecolor{colorProposition}{HTML}{FF5733}
\definecolor{colorProof}{HTML}{621055}
\definecolor{colorDefinition}{HTML}{FF9F45}
\definecolor{colorAxiom}{HTML}{506D84}
\definecolor{colorProcedure}{HTML}{753422}
\definecolor{colorProblem}{HTML}{374045}
\definecolor{colorObservation}{HTML}{EA9ABB}
\definecolor{colorABM}{HTML}{89B5AF}
\definecolor{colorExample}{HTML}{97C4B8}
\definecolor{colorConjecture}{RGB}{217, 170, 76}


\newcommand{\EXP}{\mathbb{E}}
\newcommand{\maj}{{\textit{maj}}}
\renewcommand{\epsilon}{\varepsilon}


\begin{document}
    %% Model
    % \begin{table}
    %     \begin{tabular}{rl}
    %         agents & \(1, 2, \dots , n\) \\
    %         [0.4em]
    %         time & \(t \in \{0, 1, 2, \dots \} \) \\
    %         [0.4em]
    %         true state & \(\mu \in (0, 1)\) \\
    %         [0.4em]
    %         belief of agent \(i\) at \(t\) & number between \(0\) and \(1\)\\
    %                                        & drawn from a distribution with mean \(\mu\) \\
    %                                        & and finite variance above a threshold \(\delta > 0\)\\
    %         [0.4em]
    %         social network & aperiodic, strongly connected directed graph with agents as vertices, \\ 
    %                        & and who-pays-attention-to-who as edges \\
    %         [0.4em]
    %         agent \(i\)'s neighborhood & agents that \(i\) pays attention to \\
    %         [0.4em]
    %         weight on edge from \(i\) to \(j\) & number that indicates how much weight \(i\) places on \(j\)'s opinion;\\ 
    %                                            & we assume \(i\) distributes a total weight of \(1\) across \(i\)'s neighborhood\\
    %         [0.4em]
    %         update rule & at time \(t+1\) every agent updates their belief \\ 
    %                     & to a weighted average over the beliefs of neighbors
    %     \end{tabular}
    % \end{table}

    % \( t = \infty\)

    % We write \(G_n\) for a network with \(n\) vertices.\\

    % A sequence \(G_1, G_2, \dots , G_n, \dots \) of (strongly connected and aperiodic) networks of increasing size 
    % is \emph{wise} if the consensus belief {approaches} the true state \(\mu\) asymptotically, as \(n\) goes to infinity.

    % A sequence \(G_1, G_2, \dots , G_n, \dots \) of (strongly connected and aperiodic) networks of increasing size 
    % is \emph{wise} if and only if the eigenvector centrality of every agent \(i\) approaches \(0\) asymptotically, 
    % as \(n\) goes to infinity.

    % The eigenvector centralities are \(\bm{c} = (\nicefrac{2}{3}, \nicefrac{1}{6}, \nicefrac{1}{6})\).\\

    % Centralities indicate the importance of the nodes for \\the limit consensus belief:
    % \begin{flalign*}
    %     \left(\frac{2}{3}, \frac{1}{6}, \frac{1}{6}\right)\cdot \left(1, 0, 0\right) &= 
    %         \frac{2}{3}\cdot 1 + \frac{1}{6}\cdot 0 + \frac{1}{6}\cdot 0 &&\\ 
    %         & = \frac{2}{3}. &&
    % \end{flalign*}

    %% Model for cascades
    % \begin{table}
    %     \begin{tabular}{rl}
    %         agents & \(N = \{1, \dots , n\}\) \\
    %         alternatives & \(A = \{a,b\}\) \\
    %         true alternative & \(\theta \in A\), we usually assume \(\theta = a\) \\
    %         voter \(i\)'s signal & \(s_i \in A\)\\
    %         probability of a correct signal \(i\)'s & \(\Pr[s_i = \theta ] = p\), with \(p > \nicefrac{1}{2}\) \\
    %         agents' prior probabilities & \(\Pr\left[ \theta = a \right] = \Pr\left[ \theta = b \right] = \nicefrac{1}{2}\)\\
    %         agent \(i\)'s verdict & \(v_i \in A\) \\
    %                             & agents speak out in sequence, and see previous verdicts
    %     \end{tabular}
    % \end{table}

    % \[
    %     \Pr\left[ A | B \right] = \frac{\Pr\left[ B | A \right] \cdot \Pr\left[ A \right] }{\Pr\left[ B \right]}
    % \]

    % \begin{flalign*}
    %     \Pr\left[ \theta = a \mid s_1 = a \right] & = \frac{ \Pr\left[ s_1 = a \mid \theta = a \right] \cdot \Pr\left[ \theta = a \right]}{\Pr\left[ s_1 = a \right]}& \\
    %     \Pr\left[ \theta = b \mid s_1 = a \right] & = \frac{ \Pr\left[ s_1 = a \mid \theta = b \right] \cdot \Pr\left[ \theta = b \right]}{\Pr\left[ s_1 = a \right]}&
    % \end{flalign*}

    % With \(p > \nicefrac{1}{2}\), we have that \(\Pr\left[ \theta = a \mid s_1 = a \right] > \Pr\left[ \theta = b \mid s_1 = a \right]\).


    % \begin{flalign*}
    %     \Pr\left[ \theta = b \mid s_3 = b, v_2 = v_1 = b \right] & > \Pr\left[ \theta = a \mid s_3 = b, v_2 = v_1 = b \right] &\\
    %     \Pr\left[ \theta = b \mid s_3 = a, v_2 = v_1 = b \right] & > \Pr\left[ \theta = a \mid s_3 = a, v_2 = v_1 = b \right] &
    % \end{flalign*}

    % \(\nicefrac{1}{2(n{-}1)}\)

    The network grows by adding agents that listen to the central agent 1.\\

    The eigenvector centralities are:

    \[
        \bm{c} = \left(\frac{1}{2}, \frac{1}{2(n-1)}, \dots , \frac{1}{2(n-1)}\right)
    \]

    Agent 1 retains a constant share of (network) influence as \(n\) grows.\\

    And thus decides the consensus belief.\\

    No bueno.
\end{document}