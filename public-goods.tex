%% magick convert -density 1200 test.pdf test.png
\documentclass[
    preview, 
    varwidth=8cm, 
    border={0pt 1pt 1pt 1pt}
    ]{standalone} % border options are {left bottom right top}


% \usepackage[T1]{fontenc}
% \usepackage[sfdefault]{AlegreyaSans}
% \renewcommand*\oldstylenums[1]{{\AlegreyaSansOsF #1}}

% \usepackage[sfdefault]{FiraSans} %% option 'sfdefault' activates Fira Sans as the default text font
% \usepackage[T1]{fontenc}
% \renewcommand*\oldstylenums[1]{{\firaoldstyle #1}}

\usepackage[T1]{fontenc}
\usepackage{Alegreya} %% Option 'black' gives heavier bold face 
\renewcommand*\oldstylenums[1]{{\AlegreyaOsF #1}}


% \usepackage[default]{lato}
% \usepackage[T1]{fontenc}

% \usepackage[sfdefault]{cabin}
% \usepackage[T1]{fontenc}

% \usepackage[sfdefault]{roboto}  %% Option 'sfdefault' only if the base font of the document is to be sans serif
% \usepackage[T1]{fontenc}

% \usepackage[sfdefault]{noto}
% \usepackage[T1]{fontenc}

% \usepackage[T1]{fontenc}
% \usepackage[sfdefault]{josefin}

\usepackage{amsthm}
\usepackage{amssymb}
\usepackage{amsmath}
\usepackage{bm}
\usepackage{nicefrac}
\usepackage{graphicx}
\usepackage{tikz}
\usepackage{booktabs}
\usepackage{soul}

\usetikzlibrary{automata, positioning}

%% Colors
% grey
\definecolor{DarkCharcoal}{RGB}{30, 30, 30} % theorem
\definecolor{Davys}{RGB}{60, 60, 60} % proof
\definecolor{SonicSilver}{RGB}{90, 90, 90} % proposition
\definecolor{Gray}{RGB}{120, 120, 120} % lemma
\definecolor{Spanish}{RGB}{150, 150, 150} % corollary
\definecolor{Argent}{RGB}{180, 180, 180} % example

% blues
\definecolor{Azure}{RGB}{0, 128, 255}
\definecolor{Egyptian}{RGB}{16, 52, 166}
\definecolor{Navy}{RGB}{17, 30, 108}
\definecolor{CeruleanFrost}{RGB}{104, 145, 195}
\definecolor{LightSteelBlue}{RGB}{170, 197, 226}

% reds
\definecolor{Renate}{RGB}{227, 25, 113}
\definecolor{Burgundy}{RGB}{141, 2, 31}
\definecolor{Salmon}{RGB}{250, 128, 114}

% yellows
\definecolor{Honey}{RGB}{235, 150, 5}

% greens
\definecolor{PastelGreen}{HTML}{6ECB63}
\definecolor{FernGreen}{HTML}{116530}
\definecolor{CadmiumGreen}{HTML}{046C41}


%% Color Schemes
% Pastel Tones Color Scheme
\definecolor{Manatee}{RGB}{154, 145, 172} % darkest
\definecolor{Lilac}{RGB}{202, 167, 189}
\definecolor{CherryBlossomPink}{RGB}{255, 185, 196}
\definecolor{LightRed}{RGB}{255, 211, 212}

% Luxury Gradient Color Scheme
\definecolor{PhilippineBronze}{RGB}{112, 54, 14}
\definecolor{GoldenBrown}{RGB}{147, 93, 35}
\definecolor{UniversityOfCaliforniaGold}{RGB}{182, 131, 55}
\definecolor{Sunray}{RGB}{217, 170, 76}
\definecolor{OrangeYellowCrayola}{RGB}{252, 208, 96}

% Breaking Apart Color Scheme
\definecolor{DeepTuscanRed}{RGB}{103, 74, 82}
\definecolor{FuzzyWuzzy}{RGB}{199, 114, 113}
\definecolor{MiddleYellowRed}{RGB}{232, 187, 109}
\definecolor{Independence}{RGB}{83, 70, 103}
\definecolor{CeladonGreen}{RGB}{47, 129, 133}

% Lilac
\definecolor{Lilac1}{HTML}{97cded} % blue
\definecolor{Lilac2}{HTML}{02a0da}
\definecolor{Lilac3}{HTML}{b8afc9} % purple
\definecolor{Lilac4}{HTML}{eae1ef}
\definecolor{Lilac5}{HTML}{eadbd7} % light coffee

% Monochrome Coffee
\definecolor{MonoCoffee1}{HTML}{281E15} % dark
\definecolor{MonoCoffee2}{HTML}{47423E}
\definecolor{MonoCoffee3}{HTML}{928477}
\definecolor{MonoCoffee4}{HTML}{B7B2AC}
\definecolor{MonoCoffee5}{HTML}{A49FA3} % light

% Monochrome Amethyst
\definecolor{MonoAmethyst1}{HTML}{22222E} % dark
\definecolor{MonoAmethyst2}{HTML}{393A5A}
\definecolor{MonoAmethyst3}{HTML}{706F8E}
\definecolor{MonoAmethyst4}{HTML}{ADA9BA}
\definecolor{MonoAmethyst5}{HTML}{E9E9E9} % light

% Warm Brown
\definecolor{WarmBrown1}{HTML}{3C0906} % dark
\definecolor{WarmBrown2}{HTML}{84352D}
\definecolor{WarmBrown3}{HTML}{BC5F41}
\definecolor{WarmBrown4}{HTML}{E4E4E6}
\definecolor{WarmBrown5}{HTML}{E4C08A} % light

% FMAS course
\definecolor{GameTheory}{HTML}{AA2B1D}
\definecolor{Voting}{HTML}{FE9801}
\definecolor{Matching}{HTML}{F9B384}
\definecolor{Auctions}{HTML}{583D72}
\definecolor{ABMs}{HTML}{374045}
\definecolor{WisdomOfCrowds}{HTML}{7189BF}

% environments
\definecolor{colorTheorem}{HTML}{DA2D2D}
\definecolor{colorProposition}{HTML}{FF5733}
\definecolor{colorProof}{HTML}{621055}
\definecolor{colorDefinition}{HTML}{FF9F45}
\definecolor{colorAxiom}{HTML}{506D84}
\definecolor{colorProcedure}{HTML}{753422}
\definecolor{colorProblem}{HTML}{374045}
\definecolor{colorObservation}{HTML}{EA9ABB}
\definecolor{colorABM}{HTML}{89B5AF}
\definecolor{colorExample}{HTML}{97C4B8}
\definecolor{colorConjecture}{RGB}{217, 170, 76}


\newcommand{\EXP}{\mathbb{E}}
\newcommand{\maj}{{\textit{maj}}}
\renewcommand{\epsilon}{\varepsilon}


\begin{document}\raggedright
    % \begin{table}
    %     \begin{tabular}{rl}
    %         players              & \(N = \{1, \dots, n\}\) \\
    %         \(i\)'s contribution & \(c_i \in \{0, 1\}\) \\
    %         benefit              & \(b(\bm{c}) \in \mathbb{R}\) \\
    %         \(i\)'s utility      & \(u_i(\bm{c}) = b(\bm{c}) - c_i\) \\
    %     \end{tabular}
    % \end{table}



    % A group \(N = \{1, \dots, n\}\) of \(n\) \emph{players}. 
    % Each player \(i\) makes a \emph{contribution} \(c_i \in \{0, 1\}\).
    % Contributions generate a \emph{benefit} \(b(\bm{c})\), with \(\bm{c} = (c_1, \dots, c_n)\).\\

    % Player \(i\)'s \emph{utility} is:
    % \[
    %     u_i(\bm{c}) = b (\bm{c}) - c_i.
    % \]
    % % The \emph{social welfare} of the group is:
    % % \[
    % %     \mathit{sw}(\bm{c}) = \sum_{i} u_i(\bm{c}).
    % % \]

    % The public good, valued at \(r > 1\), is provided just in case the fraction of contributors is at least 
    % a certain \emph{threshold} \({\theta \in [0, 1]}\):
    % \[
    %     b (\bm{c}) =
    %     \begin{cases}
    %         r  & \text{if the fraction of contributors in } \bm{c} \text{ is} \geq \theta,\\ 
    %         0, & \text{otherwise}.
    %     \end{cases}  
    % \]


    % Take \(\theta = \nicefrac{1}{2}\)\\
    
    % At least three contributors are needed for the public good to be provided.\\

    % A player has an incentive to contribute only if they are \emph{pivotal}.


    

    % In experimental settings, the benefit function is often:
    % \[
    %     b(\bm{c}) = \frac{1}{n} \cdot \sum_{i}c_i \cdot g,
    % \]
    % for some \(g > 1\).\\

    % In this setting, contributing \(0\) is a dominant strategy.\\ 

    % But we get the same equilibria (with \(0\) and \(k\) contributors) 
    % if the game is repeated for \(m > \nicefrac{(n-1)}{n} \cdot \nicefrac{r}{(r-1)}\) rounds
    % and players play a threshold \emph{strategy}. 

    

    % For a threshold of \(k\),\(^*\) profiles with exactly \(k\) contributors are equilibria.\\
    
    % The profile with \(0\) contributors is also an equilibrium (except when \(k=1\)).

    % \begin{flushright}
    %     \(^*\)Every threshold \(\theta\) corresponds to an integer threshold of \(k \in \{0,\dots, n\}\).
    % \end{flushright}


    % Take the threshold to be \(\theta = \nicefrac{1}{2}\), i.e., 
    % a player benefits only if at least half of their closed neighborhood contribute.\\ 
    
    % A profile where everyone contributes is still not an equilibrium.\\

    % The profile with \(0\) contributors is no longer an equilibrium (!).\\
    
    % The profile where only the center contributes is not an equilibrium.\\

    % But the profile where only the agents on the periphery contribute is an equilibrium.


    % On average, we get around \(qn\) initial contributors.


    % Assume the game is played over several rounds.\\

    % At every new round, player \(i\) contributes their \emph{best response} 
    % \(\mathit{br}_i(\bm{c})\) with respect to the previous contributions \(\bm{c}\):
    % \[
    %     \mathit{br}_i(\bm{c}) = 
    %     \begin{cases}
    %         1, &\text{if } i \text{ is pivotal in } \bm{c},\\
    %         0, &\text{otherwise}.            
    %     \end{cases}
    % \]



    % If exactly three (or two!) players contribute initially, 
    % we reach an equilibrium with three contributors.\\

    % Anything else and we end up in the \((0, \dots, 0)\) equilibrium.


    % At every new round \(t+1\), player \(i\) contributes:
    % \[
    %     c_i ^{t+1} = 
    %     \begin{cases}
    %         \mathit{br}_i(\bm{c}^t), &\text{with probability } 1-\alpha, \\ 
    %         c_i^t, &\text{with probability } \alpha.
    %     \end{cases}
    % \]


    % With \(\alpha = 1\), players just repeat their initial contribution.\\

    % With \(\alpha = 0\), players invariably play their best response.\\

    % For \(0 < \alpha < 1\), players play a mixture.\\

    % The equilibrium at \(\theta\) is reached, most of the time, if 
    % \(q > \theta\).\\ 
    
    % But only for very high \(\alpha\)!


    % A \emph{social norm applied to player \(i\)} is a function 
    % \(\nu \colon N \times \{0, 1\}^n \rightarrow \{0, 1\}\):
    % \[
    %     (i, \bm{c}) \longmapsto  \nu_i (\bm{c})
    % \]




    % Take \(\alpha \in [0, 1]\), the \emph{prosociality} of players.\\

    % At every new round \(t+1\), player \(i\) contributes:
    % \[
    %     c_i ^{t+1} = 
    %     \begin{cases}
    %         \nu_i(\bm{c}^t), &\text{with probability } \alpha,\\
    %         \mathit{br}_i(\bm{c}^t), &\text{with probability } 1-\alpha.
    %     \end{cases}
    % \]



    % \begin{tabular}{rl}
    %     Best Response: & \(\nu_i(\bm{c}) = \mathit{br}_i(\bm{c})\) \\[0.2cm]
    %     Be Yourself: & \(\nu_i(\bm{c}) = c_i\)  \\[0.2cm] 
    %     Be Like Others: & 
    %     \(\nu_i(\bm{c}) = \text{the majority contribution in } \bm{c}_{-i}\) \\ 
    %     \dots   & \dots
    % \end{tabular}


    % The majority rule ('Be Like Others') does a bit better, at lower values of 
    % prosociality.\\ 

    % But we still need \(q > \theta\)!

    % So far we have been assuming that everyone is connected to everyone else.\\ 

    % But many other structures are possible.



    % On networks, \(i\)'s utility is:
    % \[
    %     u_i(\bm{c}) = b{\left( \bm{c}_{N[i]} \right)} - c_i,
    % \]
    % where \(N[i]\) contains \(i\)'s neighbors, plus \(i\).\\


    % Player \(i\) wants to contribute only if they are pivotal
    % \emph{in their neighborhood} \(N[i]\).



    % Take \(\theta = \nicefrac{1}{2}\) and a star graph.\\ 

    % Everyone contributing still not an equilibrium.\\ 

    % No contributors is not an equilibrium (!).\\ 

    % Only the players on the periphery contributing is an equilibrium.


    % Consider the \emph{Be Like Others} norm, with \(\alpha=1\).\(^*\)\\ 

    % If the two players in the center start out cooperating,
    % we eventually get full cooperation!\\ 

    % And this generalizes to larger graphs.

    % \begin{flushright}
    %     \(^*\)Assume tie-breaking favors contributing.
    % \end{flushright}

    % The more structured the graph, the better.

    % Scale-free networks, generated using a preferential attachment mechanism.

    % Social norms as internalized behavior rules (requirements).\\ 

    % Powerful, and even more so in the presence of social networks.




    % There is a set \(N = \{1, \dots, n\}\) of \emph{players}.
    
    % \vspace{1em}
    % Each player \(i\) starts with an \emph{endowment} \(e\)
    % and makes a \emph{contribution} \(c_i \in [0, e]\).

    % \vspace{1em}
    % The \emph{public good} is obtained by pooling all contributions,
    % and multiplying them by a constant \(r > 1\).

    % \vspace{1em}    
    % Each player receives an equal share of the public good.


    % If player \(i\) contributes \(c_i > 0\), they make:
    % \begin{flalign*}
    %     \qquad
    %     e - c_i + r \cdot \frac{\sum_{j \neq i}c_j + c_i}{n}. &&
    % \end{flalign*}
    % If player \(i\) lowers their contribution to \(c'_i < c_i\), while 
    % everyone else keeps their contribution constant, they make:
    % \begin{flalign*}
    %     \qquad
    %     e - c'_i + r \cdot \frac{\sum_{j \neq i}c_j + c'_i}{n}. &&
    % \end{flalign*}
    % Lowering the contribution is worth it just in case:
    % \begin{flalign*}
    %     \qquad
    %     e - c'_i + r \cdot \frac{\sum_{j \neq i}c_j + c'_i}{n} &> e - c_i + r \cdot \frac{\sum_{j \neq i}c_j + c_i}{n} & \text{iff} \\
    %     c_i - c'_i & > r \cdot \frac{c_i-c'_i}{n} &\text{iff} \\
    %     n & > r.&
    % \end{flalign*}

    If everyone contributes \(0\), there is no public good and each player 
    is stuck with their initial endowment \(e\).\\

    \vspace{1em}
    In this situation, if a player \(i\) decides to deviate and contribute \(c_i > 0\), they make:
    \begin{flalign*}
        \qquad
        e - c_i + r \cdot \frac{c_i}{n} & < e, & \text{as long as } r < n. &&
    \end{flalign*}

\end{document}