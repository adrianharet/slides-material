%% magick convert -density 1200 test.pdf test.png
\documentclass[preview, border={0pt 5pt 3pt 1pt}, varwidth=14cm]{standalone} % border options are {left bottom right top}

% \usepackage[T1]{fontenc}
% \usepackage[sfdefault]{AlegreyaSans}
% \renewcommand*\oldstylenums[1]{{\AlegreyaSansOsF #1}}

% \usepackage[sfdefault]{FiraSans} %% option 'sfdefault' activates Fira Sans as the default text font
% \usepackage[T1]{fontenc}
% \renewcommand*\oldstylenums[1]{{\firaoldstyle #1}}

\usepackage[T1]{fontenc}
\usepackage{Alegreya} %% Option 'black' gives heavier bold face 
\renewcommand*\oldstylenums[1]{{\AlegreyaOsF #1}}


% \usepackage[default]{lato}
% \usepackage[T1]{fontenc}

% \usepackage[sfdefault]{cabin}
% \usepackage[T1]{fontenc}

% \usepackage[sfdefault]{roboto}  %% Option 'sfdefault' only if the base font of the document is to be sans serif
% \usepackage[T1]{fontenc}

% \usepackage[sfdefault]{noto}
% \usepackage[T1]{fontenc}

% \usepackage[T1]{fontenc}
% \usepackage[sfdefault]{josefin}

\usepackage{amsthm}
\usepackage{amssymb}
\usepackage{amsmath}
\usepackage{bm}
\usepackage{nicefrac}
\usepackage{graphicx}
\usepackage{tikz}
\usepackage{booktabs}
\usepackage{soul}

\usetikzlibrary{automata, positioning}

%% Colors
% grey
\definecolor{DarkCharcoal}{RGB}{30, 30, 30} % theorem
\definecolor{Davys}{RGB}{60, 60, 60} % proof
\definecolor{SonicSilver}{RGB}{90, 90, 90} % proposition
\definecolor{Gray}{RGB}{120, 120, 120} % lemma
\definecolor{Spanish}{RGB}{150, 150, 150} % corollary
\definecolor{Argent}{RGB}{180, 180, 180} % example

% blues
\definecolor{Azure}{RGB}{0, 128, 255}
\definecolor{Egyptian}{RGB}{16, 52, 166}
\definecolor{Navy}{RGB}{17, 30, 108}
\definecolor{CeruleanFrost}{RGB}{104, 145, 195}
\definecolor{LightSteelBlue}{RGB}{170, 197, 226}

% reds
\definecolor{Renate}{RGB}{227, 25, 113}
\definecolor{Burgundy}{RGB}{141, 2, 31}
\definecolor{Salmon}{RGB}{250, 128, 114}

% yellows
\definecolor{Honey}{RGB}{235, 150, 5}

% greens
\definecolor{PastelGreen}{HTML}{6ECB63}
\definecolor{FernGreen}{HTML}{116530}
\definecolor{CadmiumGreen}{HTML}{046C41}


%% Color Schemes
% Pastel Tones Color Scheme
\definecolor{Manatee}{RGB}{154, 145, 172} % darkest
\definecolor{Lilac}{RGB}{202, 167, 189}
\definecolor{CherryBlossomPink}{RGB}{255, 185, 196}
\definecolor{LightRed}{RGB}{255, 211, 212}

% Luxury Gradient Color Scheme
\definecolor{PhilippineBronze}{RGB}{112, 54, 14}
\definecolor{GoldenBrown}{RGB}{147, 93, 35}
\definecolor{UniversityOfCaliforniaGold}{RGB}{182, 131, 55}
\definecolor{Sunray}{RGB}{217, 170, 76}
\definecolor{OrangeYellowCrayola}{RGB}{252, 208, 96}

% Breaking Apart Color Scheme
\definecolor{DeepTuscanRed}{RGB}{103, 74, 82}
\definecolor{FuzzyWuzzy}{RGB}{199, 114, 113}
\definecolor{MiddleYellowRed}{RGB}{232, 187, 109}
\definecolor{Independence}{RGB}{83, 70, 103}
\definecolor{CeladonGreen}{RGB}{47, 129, 133}

% Lilac
\definecolor{Lilac1}{HTML}{97cded} % blue
\definecolor{Lilac2}{HTML}{02a0da}
\definecolor{Lilac3}{HTML}{b8afc9} % purple
\definecolor{Lilac4}{HTML}{eae1ef}
\definecolor{Lilac5}{HTML}{eadbd7} % light coffee

% Monochrome Coffee
\definecolor{MonoCoffee1}{HTML}{281E15} % dark
\definecolor{MonoCoffee2}{HTML}{47423E}
\definecolor{MonoCoffee3}{HTML}{928477}
\definecolor{MonoCoffee4}{HTML}{B7B2AC}
\definecolor{MonoCoffee5}{HTML}{A49FA3} % light

% Monochrome Amethyst
\definecolor{MonoAmethyst1}{HTML}{22222E} % dark
\definecolor{MonoAmethyst2}{HTML}{393A5A}
\definecolor{MonoAmethyst3}{HTML}{706F8E}
\definecolor{MonoAmethyst4}{HTML}{ADA9BA}
\definecolor{MonoAmethyst5}{HTML}{E9E9E9} % light

% Warm Brown
\definecolor{WarmBrown1}{HTML}{3C0906} % dark
\definecolor{WarmBrown2}{HTML}{84352D}
\definecolor{WarmBrown3}{HTML}{BC5F41}
\definecolor{WarmBrown4}{HTML}{E4E4E6}
\definecolor{WarmBrown5}{HTML}{E4C08A} % light

% FMAS course
\definecolor{GameTheory}{HTML}{AA2B1D}
\definecolor{Voting}{HTML}{FE9801}
\definecolor{Matching}{HTML}{F9B384}
\definecolor{Auctions}{HTML}{583D72}
\definecolor{ABMs}{HTML}{374045}
\definecolor{WisdomOfCrowds}{HTML}{7189BF}

% environments
\definecolor{colorTheorem}{HTML}{DA2D2D}
\definecolor{colorProposition}{HTML}{FF5733}
\definecolor{colorProof}{HTML}{621055}
\definecolor{colorDefinition}{HTML}{FF9F45}
\definecolor{colorAxiom}{HTML}{506D84}
\definecolor{colorProcedure}{HTML}{753422}
\definecolor{colorProblem}{HTML}{374045}
\definecolor{colorObservation}{HTML}{EA9ABB}
\definecolor{colorABM}{HTML}{89B5AF}
\definecolor{colorExample}{HTML}{97C4B8}
\definecolor{colorConjecture}{RGB}{217, 170, 76}


\newcommand{\EXP}{\mathbb{E}}
\newcommand{\maj}{{\textit{maj}}}
\renewcommand{\epsilon}{\varepsilon}


\begin{document}
    %% Model
    % \begin{table}
    %     \begin{tabular}{rl}
    %         voters & \(N = \{1, \dots , n\}\) \\
    %         alternatives & \(A = \{a,b\}\) \\
    %         correct alternative & \(a\) \\
    %         voter \(i\)'s vote & \(v_i \in A\) \\
    %         profile of votes & \(\bm{v} = (v_1, \dots , v_n)\) \\
    %         voter \(i\)'s competence & \(\Pr[v_i = a] = p_i\), with \(p_i \in [0,1]\) \\
    %         majority vote & \(F_{\maj}(\bm{v}) = x\), such that \(v_i = x\) for a (strict) majority of voters
    %     \end{tabular}
    % \end{table}

    % % Notation
    % There is a set \(N = \{1, \dots, n\}\) of \emph{voters}. 
    % There are two \emph{alternatives}, \(a\) and \(b\), one of which is \emph{correct}.
    % Each voter casts a \emph{vote} for one of the alternatives. We keep track of whether each voter \(i\) is 
    % correct using a \emph{random variable} \(v_i\):
    % \[
    %     v_i = \begin{cases}
    %         1, & \text{if voter } i \text{ votes for the correct alternative,} \\
    %         0, & \text{otherwise.}
    %     \end{cases}
    % \]
    % The \emph{profile} of votes is a vector \(\bm{v} = (v_1, \dots, v_n)\) of the votes cast.
    % The \emph{majority outcome}\(^*\) is the alternative with the most votes.\\

    % Each voter \(i\) has a \emph{competence} \(p_i\), which is their probability of voting correctly:
    % \[
    %     v_i = 
    %     \begin{cases}
    %         1, & \text{with probability } p_i, \\
    %         0, & \text{with probability } 1 - p_i.
    %     \end{cases}
    % \]

    % \footnotesize \(^*\)We assume \(n\) is odd to avoid ties.


    % May I humbly point out that the vote random variables \(v_i\) 
    % are called \emph{Bernoulli variables}: \(v_i = 1\) is \emph{success},
    % \(v_i = 0\) is \emph{failure}.\\

    % The sum of the votes is also a random variable:
    % \[
    %     S_n = v_1 + \dots + v_n.
    % \]
    % \(S_n\) tracks the number of correct votes in a profile of \(n\) votes.\\

    % Note that the majority outcome is correct exactly when 
    % \({S_n > \left\lfloor \nicefrac{n}{2} \right\rfloor}\).




    % \({\Pr}\Big[S_n > \lfloor \nicefrac{n}{2} \rfloor \Big]\)



    %% Assumptions
    % \begin{description}
    %     \item[(Competence)] Agents are better than random at being correct:
    %     \[
    %         p_i > \frac{1}{2},~\text{for any voter}~i \in N.
    %     \]
    %     \item[(Equal Competence)] All agents have the same competence:
    %     \[
    %         p_i = p_j = p,~\text{for all voters}~i,j \in N.
    %     \]
    %     \item[(Independence)] Voters vote independently of each other:
    %     \[
    %         \Pr[v_i = x, v_j=y] = \Pr[v_i=x] \cdot \Pr[v_j=y],~\text{for all voters}~i,j \in N.
    %     \]
    % \end{description}
    % \vspace{0.1em}


    % \({\Pr}{\left[ F_\maj(\bm{v}) = 1\right]}\)



    % % The Condorcet Jury Theorem
    % For \(n\) voters with equal competence \(p > \nicefrac{1}{2}\) that vote independently of each other,
    % then, for any odd \(n\), it holds that:
    % \begin{itemize}
    %     \item[(1)] \({\Pr}{\Big[ S_{n+2} > \lfloor \nicefrac{(n+2)}{2} \rfloor\Big]} > 
    %         {\Pr}{\Big[ S_n > \lfloor \nicefrac{n}{2} \rfloor\Big]}\), and
    %     \item[(2)] \({\Pr}{\Big[ S_n > \lfloor \nicefrac{n}{2} \rfloor \Big]} \geq p\), and
    %     \item[(3)] \(\lim_{n\rightarrow\infty}{\Pr}{\Big[ S_n > \lfloor \nicefrac{n}{2} \rfloor \Big]} = 1\).
    % \end{itemize}


    %% CJT explanations:
    % \begin{itemize}
    %     \item[(1)] Larger groups are more accurate than smaller groups.

    %     \item[(2)] Groups are more accurate than their members.

    %     \item[(3)] The probability of a correct decision approaches \(1\) as the group size increases.
    % \end{itemize}


    % One voter
    % The profile is \(\bm{v} = (v_1)\).\\
    
    % The probability of a correct decision is:
    % \begin{flalign*}
    %     {\Pr}{\Big[S_1 > 0\Big]} & = {\Pr}{\Big[v_1 = 1\Big]} &\\
    %                                     & = p &\\
    %                                     & > \nicefrac{1}{2}.&
    % \end{flalign*}
    % As \(p\) grows, so does group accuracy.


    %% Two voters
    % The profile is \(\bm{v} = (v_1, v_2) \).\\

    % Oh wait, we're not looking at this case.


    % Three voters
    % The profile is \(\bm{v} = (v_1, v_2, v_3)\).\\

    % The probability of a correct decision is:
    % \begin{flalign*}
    %     {\Pr}{\Big[S_3 > 1\Big]} 
    %         & = {\Pr}{\Big[S_3 = 2 \text{ or } S_3 = 3\Big]} &\\
    %         & = {\Pr}{\Big[\bm{v} \text{ is one of } (1,1,0), (1,0,1), (0,1,1) \text{ or } (1,1,1)\Big]} &\\
    %         & = {\Pr}{\Big[\bm{v} = (1, 1, 0)\Big]} + 
    %             {\Pr}{\Big[\bm{v} = (1, 0, 1)\Big]} +
    %             {\Pr}{\Big[\bm{v} = (0, 1, 1)\Big]} +
    %             {\Pr}{\Big[\bm{v} = (1, 1, 1)\Big]}
    %             &\\
    %         & = {\Pr}{\big[v_1 = 1\big]} \cdot {\Pr}{\big[v_2 = 1\big]} \cdot {\Pr}{\big[v_3 = 0\big]} + \\
    %         & \qquad {\Pr}{\big[v_1 = 1\big]} \cdot {\Pr}{\big[v_2 = 0\big]} \cdot {\Pr}{\big[v_3 = 1\big]} + \\
    %         & \qquad\quad {\Pr}{\big[v_1 = 0\big]} \cdot {\Pr}{\big[v_2 = 1\big]} \cdot {\Pr}{\big[v_3 = 1\big]} + \\
    %         & \qquad\qquad {\Pr}{\big[v_1 = 1\big]} \cdot {\Pr}{\big[v_2 = 1\big]} \cdot {\Pr}{\big[v_3 = 1\big]} &\\
    %         & = p \cdot p \cdot (1-p) + p \cdot(1-p) \cdot p + (1-p) \cdot p \cdot p + p \cdot p \cdot p &\\
    %         & = 3p^2(1-p) + p^3 &\\
    %         & > p.&
    % \end{flalign*}
    % Again, as \(p\) grows, so does group accuracy.\\ 

    % And a group of three voters is more accurate than a single voter!


    % % Five voters
    % The profile is \(\bm{v} = (v_1, v_2, v_3, v_4, v_5)\).\\

    % The probability of a correct decision is:
    % \begin{flalign*}
    %     {\Pr}{\Big[S_5 > 2 \Big]} 
    %         & = {\Pr}{\Big[S_5 = 3 \text{ or } S_5 = 4 \text{ or } S_5 = 5 \Big]} &\\
    %                     & = {\Pr}{\big[\bm{v} \text{ is either } (1,1,1,0,0), \dots, (1,1,1,1,0), \dots, \text{ or } (1,1,1,1,1)\big]} &\\
    %                     & \dots &\\
    %                     & = 10 \cdot p^3(1-p)^2 + 5 \cdot p^4(1-p) + p^5 &\\
    %                     & = {5 \choose 3} p^3(1-p)^2 + {5 \choose 4} p^4(1-p) + {5 \choose 5} p^5 &
    % \end{flalign*}
    % Again, as \(p\) grows, so does group accuracy.\\ 

    % And a group of five voters is more accurate than a group of three!



    % % Any odd number of voters
    % The profile is \(\bm{v} = (v_1, \dots, v_n)\), for \(n=2k+1\) and \(k \geq 1\).\\

    % The probability of a correct decision is:
    % \begin{flalign*}
    %     {\Pr}{\Big[S_n > k \Big]} 
    %         & = {\Pr}{\Big[S_n = k{+}1 \text{ or } \dots \text{ or } S_n = n\Big]} &\\
    %         & = {n \choose k+1} \cdot p^{k+1}(1-p)^{n-(k+1)} + \dots + {n \choose n-1} \cdot p^{n-1}(1-p)^1 + {n \choose n} p^n & \\ 
    %         & = \sum_{i=k+1}^{n} {n \choose i} \cdot p^i(1-p)^{n-i}.
    % \end{flalign*}
    % And it looks like the same reasoning applies: as \(n\) grows, so does group accuracy!\\

    % But only as long as \(p > \nicefrac{1}{2}\)\dots




    % To prove that accuracy increases with group size, we derive a recurrence relation for the probability of a correct decision
    % with \(n+2\) voters, given the probability of a correct decision with \(n\) voters.\\

    % Take \(n=5\).

    % \[
    %     {\Pr}{\Big[ S_5 > 2 \Big]} = 
    %         (1-p)^2 \cdot {\Pr}{\Big[ S_3 > 2 \Big]} +
    %         2p(1-p)^2 \cdot {\Pr}{\Big[ S_3 > 1 \Big]} +
    %         p^2 \cdot {\Pr}{\Big[ S_3 > 0 \Big]}.
    % \]


    % \[
    %     \textcolor{GameTheory}{{\Pr}{\left[ S_{n+2} > \left\lfloor\frac{n+2}{2} \right\rfloor \right]}} = 
    %     (1-p)^2 \cdot \textcolor{Azure}{{\Pr}{\left[ S_n > \left\lfloor\frac{n}{2}\right\rfloor + 1 \right]}} + 
    %     2p(1-p)^2 \cdot \textcolor{Honey}{{\Pr}{\left[ S_n > \left\lfloor \frac{n}{2}\right\rfloor \right]}} + 
    %     p^2 \cdot \textcolor{PastelGreen}{{\Pr}{\left[ S_n > \left\lfloor \frac{n}{2} \right\rfloor - 1 \right]}}        
    % \]

    % \[
    %     \textcolor{GameTheory}{{\Pr}{\Big[ S_{2k+3} > k{+}1 \Big]}} = 
    %     (1-p)^2 \cdot \textcolor{Azure}{{\Pr}{\Big[ S_{2k+1} > k {+} 1 \Big]}} + 
    %     2p(1-p)^2 \cdot \textcolor{Honey}{{\Pr}{\Big[ S_{2k+1} > k \Big]}} + 
    %     p^2 \cdot \textcolor{PastelGreen}{{\Pr}{\Big[ S_{2k+1} > k {-} 1 \Big]}}.
    % \]


    % \begin{align*}
    %     \textcolor{PastelGreen}{{\Pr}{\Big[ S_{2k+1} > k{-}1 \Big]}} 
    %         & = \textcolor{Honey}{{\Pr}{\Big[ S_{2k+1} > k \Big]}} + {\Pr}{\Big[ S_{2k+1} = k \Big]} &\\
    %         & = \textcolor{Honey}{{\Pr}{\Big[ S_{2k+1} > k \Big]}} + \binom{2k{+}1}{k}\cdot p^k(1-p)^{k+1} &\\
    %     \textcolor{Azure}{{\Pr}{\Big[ S_{2k+1} > k{+}1 \Big]}} 
    %         & = \textcolor{Honey}{{\Pr}{\Big[ S_{2k+1} > k \Big]}} - {\Pr}{\Big[ S_{2k+1} = k{+}1 \Big]} &\\
    %         &= \textcolor{Honey}{{\Pr}{\Big[ S_{2k+1} > k \Big]}} - \binom{2k+1}{k+1}\cdot p^{k+1}(1-p)^{k}
    % \end{align*}

    % \[
    %     {2k+1 \choose k} = {2k+1 \choose k+1} = c
    % \]

    % \begin{align*}
    %     \textcolor{GameTheory}{{\Pr}{\Big[ S_{2k+3} > k{+}1 \Big]}} 
    %     & = (1-p)^2 \cdot \textcolor{Azure}{{\Pr}{\Big[ S_{2k+1} > k {+} 1 \Big]}} + 
    %         2p(1-p)^2 \cdot \textcolor{Honey}{{\Pr}{\Big[ S_{2k+1} > k \Big]}} + 
    %         p^2 \cdot \textcolor{PastelGreen}{{\Pr}{\Big[ S_{2k+1} > k {-} 1 \Big]}} \\
    %     & \dots \\
    %     & = \textcolor{Honey}{{\Pr}{\Big[ S_{2k+1} > k \Big]}} + c \cdot p^{k+1} \cdot (1-p)^{k+1} \cdot (2p-1)\\ 
    %     & > \textcolor{Honey}{{\Pr}{\Big[ S_{2k+1} > k \Big]}}.
    % \end{align*}



    % \begin{align*}
    %     \textcolor{PastelGreen}{{\Pr}{\left[ S_n > \left\lfloor \frac{n}{2}\right\rfloor-1 \right]}} &= 
    %     \textcolor{Honey}{{\Pr}{\left[ S_n > \left\lfloor\frac{n}{2} \right\rfloor \right]}} + 
    %     \binom{n}{\lfloor\nicefrac{n}{2}\rfloor}\cdot p^{\lfloor\nicefrac{n}{2}\rfloor}(1-p)^{\lfloor\nicefrac{n}{2}\rfloor+1}\\
    %     \textcolor{Azure}{{\Pr}{\left[ S_n > \left\lfloor \frac{n}{2}\right\rfloor+1 \right]}} &= 
    %     \textcolor{Honey}{{\Pr}{\left[ S_n > \left\lfloor\frac{n}{2} \right\rfloor \right]}} -
    %     \binom{n}{\lfloor\nicefrac{n}{2}+1\rfloor}\cdot p^{\lfloor\nicefrac{n}{2}\rfloor+1}(1-p)^{\lfloor\nicefrac{n}{2}\rfloor}
    % \end{align*}


    % Generalizing the previous identity we get the following recurrence:
    % {
    %     \footnotesize
    %     \[
    %         \textcolor{GameTheory}{{\Pr}{\left[ S_{n+2} > \left\lfloor\frac{n+2}{2} \right\rfloor \right]}} = 
    %         (1-p)^2 \cdot \textcolor{Azure}{{\Pr}{\left[ S_n > \left\lfloor\frac{n}{2}\right\rfloor + 1 \right]}} + 
    %         2p(1-p)^2 \cdot \textcolor{Honey}{{\Pr}{\left[ S_n > \left\lfloor \frac{n}{2}\right\rfloor \right]}} + 
    %         p^2 \cdot \textcolor{PastelGreen}{{\Pr}{\left[ S_n > \left\lfloor \frac{n}{2} \right\rfloor - 1 \right]}}        
    %     \]
    % }

    % \noindent
    % The events on the right-hand-side can be rewritten as:
    % {
    % \footnotesize
    % \begin{align*}
    %     \textcolor{PastelGreen}{{\Pr}{\left[ S_n > \left\lfloor \frac{n}{2}\right\rfloor-1 \right]}} &= 
    %     \textcolor{Honey}{{\Pr}{\left[ S_n > \left\lfloor\frac{n}{2} \right\rfloor \right]}} + 
    %     \binom{n}{\lfloor\nicefrac{n}{2}\rfloor}\cdot p^{\lfloor\nicefrac{n}{2}\rfloor}(1-p)^{\lfloor\nicefrac{n}{2}\rfloor+1}\\
    %     \textcolor{Azure}{{\Pr}{\left[ S_n > \left\lfloor \frac{n}{2}\right\rfloor+1 \right]}} &= 
    %     \textcolor{Honey}{{\Pr}{\left[ S_n > \left\lfloor\frac{n}{2} \right\rfloor \right]}} -
    %     \binom{n}{\lfloor\nicefrac{n}{2}+1\rfloor}\cdot p^{\lfloor\nicefrac{n}{2}\rfloor+1}(1-p)^{\lfloor\nicefrac{n}{2}\rfloor}
    % \end{align*}
    % }

    % \noindent
    % Plug second and third identities into first, and write \(\binom{n}{\lfloor\nicefrac{n}{2}\rfloor} = \binom{n}{\lfloor\nicefrac{n}{2}\rfloor+1}=c\):
    % {
    %     \footnotesize
    %     \[
    %         \textcolor{GameTheory}{{\Pr}{\left[ S_{n+2} > \left\lfloor\frac{n+2}{2} \right\rfloor \right]}} = 
    %         \textcolor{Honey}{{\Pr}{\left[ S_n > \left\lfloor \frac{n}{2}\right\rfloor \right]}} + 
    %         c \cdot p^{\lfloor\nicefrac{n}{2}\rfloor+1}(1-p)^{\lfloor\nicefrac{n}{2}\rfloor+1}(2p-1).
    %     \]
    % }

    % \noindent
    % Since \(\nicefrac{1}{2}<p<1\), the second term on the right-hand side is positive.

    % This follows from Claim 1:
    % \begin{flalign*}
    %     p & = {\Pr}{\Big[ S_1 > 0 \Big]} \\ 
    %       & < {\Pr}{\Big[ S_3 > 1 \Big]} \\
    %       & \dots \\ 
    %       & < \Pr{\Big[ S_n > \lfloor \nicefrac{n}{2} \rfloor \Big]} \\ 
    %       & \dots 
    % \end{flalign*}

    %% Weak Law of Large Numbers
    % If \(X_1\), \dots, \(X_n\) are independent and identically distributed (i.i.d.) random variables
    % such that \(\EXP[X_i] = \mu\), then, for any \(\epsilon > 0\), it holds that:
    % \[
    %     \lim_{n \rightarrow \infty} \Pr\left[ \left|\frac{X_1 + \dots + X_n}{n} - \mu\right|< \epsilon \right] = 1.
    % \]

  % In our case, each independent random variable \(v_i\) 
    % keeps track of whether voter \(i\) votes correctly, with:
    % \[
    %     v_i = \begin{cases}
    %         1, & \text{with probability } p \\
    %         0, & \text{with probability } 1-p.
    %     \end{cases}
    % \]
    % The majority vote is correct when:
    % \[
    %     v_1 + v_2 + \dots + v_n > \frac{n}{2} \quad\text{iff}\quad
    %     \frac{v_1 + \dots v_n}{n} > \frac{1}{2}.
    % \]


    % The \emph{Law of Large Numbers} gives us that, as \(n\) grows, \(\nicefrac{(v_1 + \dots v_n)}{n}\)
    % gets very close to the expected value of the random variables \(v_i\).\\

    % The expected value (i.e., mean \(\mu\)) is:
    % \begin{flalign*}
    %     && \EXP\big[v_i\big] & = 1 \cdot p + 0 \cdot (1-p) &\\ 
    %     &&           & = p.&
    % \end{flalign*}
    % So, for very large \(n\), with high probability:
    % \begin{flalign*}
    %     && \frac{v_1 + \dots + v_n}{n} & \approx p &\\ 
    %     &&                             & > \frac{1}{2}.
    % \end{flalign*}
    % This can be made precise with an appropriate choice of \(\epsilon\) 
    % in the \emph{Law of Large Numbers}.
    





    % Take three voters with competences \(p_1\), \(p_2\), \(p_3\).\\ 

    % The probability of a correct majority decision is:
    % \begin{flalign*}
    %     \quad{\Pr}\Big[ S_n > 1 \Big] & = {\Pr}{\Big[S_n = 2 \text{ or } S_n = 3 \Big]} &\\
    %                                   & = p_1p_2(1-p_3) + p_1(1-p_2)p_3 + (1-p_1)p_2p_3 + p_1p_2p_3 &\\
    %                                   & = p_1p_2 + p_2p_3 + p_1p_3 - 2p_1p_2p_3.&
    % \end{flalign*}
    % For \(n\) voters, \(n\) odd, with competences \(p_1\), \dots, \(p_n\), the probability of a correct majority decision is:
    % \begin{flalign*}
    %     \quad{\Pr}\Big[ S_n > \nicefrac{n}{2} \Big] & = 
    %         \sum_{C \subseteq N, |C| > \nicefrac{n}{2}} \left(\prod_{i \in C} p_i \cdot \prod_{N \setminus C}(1-p_i) \right).&  
    % \end{flalign*}





    % Take three voters with competences \(p_1 = 1\), \(p_2 = \nicefrac{3}{4}\) and \(p_3 = \nicefrac{5}{8}\).\\

    % The probability of a correct majority decision is:
    % \begin{flalign*}
    %     \quad{\Pr}\Big[ S_3 > 1 \Big] & = {\Pr}{\Big[S_3 = 2 \text{ or } S_3 = 3 \Big]} &\\
    %                                   & = p_1p_2(1-p_3) + p_1(1-p_2)p_3 + (1-p_1)p_2p_3 + p_1p_2p_3 &\\
    %                                   & = 1 \cdot \frac{3}{4} \cdot \left(1-\frac{5}{8}\right) + 1 \cdot \left(1-\frac{3}{4}\right) \cdot \frac{5}{8} + (1-1) \cdot \frac{3}{4} \cdot \frac{5}{8} + 1 \cdot \frac{3}{4} \cdot \frac{5}{8} &\\
    %                                   & = 0.90625& \\
    %                                   & < p_1.&
    % \end{flalign*}
    % Now we have an expert (i.e., voter \(1\)) who is actually better than the majority vote.





    % Suppose we add two voters with competences \(p_4 = \nicefrac{9}{16}\) and \(p_5 = \nicefrac{17}{32}\)
    % to the previous group.\\

    % We now have a group of five voters whose competences are
    % \({\bm{p} = \Big(1, \nicefrac{3}{4}, \nicefrac{5}{8}, \nicefrac{15}{16}, \nicefrac{16}{32} \Big)}\).\\

    % The probability of a correct majority decision is:
    % \begin{flalign*}
    %     \quad{\Pr}\Big[ S_5 > 2 \Big] & \approx 0.84\\ 
    %                                   & < 0.91 & \\
    %                                   & \approx {\Pr}\Big[ S_3 > 1 \Big].&
    % \end{flalign*}
    % Adding voters \(4\) and \(5\) made the group less accurate than before!



    % Take \(n\) voters with competences:
    % \begin{flalign*}
    %   \quad p_1 &= \frac{1}{2} + \frac{1}{2}, \quad p_2 = \frac{1}{2} + \frac{1}{2^2}, \quad \dots, \quad p_n = \frac{1}{2} + \frac{1}{2^n}.&      
    % \end{flalign*}
    
    % The probability of a correct majority decision, as \(n\) grows, is:
    % \begin{flalign*}
    %   \quad\lim_{n \rightarrow \infty} {\Pr}\Big[ S_n > \nicefrac{n}{2} \Big] &= \frac{1}{2}.&
    % \end{flalign*}
    % Even though the competence of each voter is above \(\nicefrac{1}{2}\), 
    % the probability of a correct majority decision does not go asymptotically towards \(1\).





    % For an odd number \(n\) of voters with competences \(p_1\), \dots , \(p_n\) 
    % who vote independently of each other, then, if \(p_n > \nicefrac{1}{2} + \epsilon \), 
    % for some \(\epsilon > 0\), it holds that:
    % \[
    %   \lim_{n\rightarrow\infty}{\Pr}\Big[ S_n > \nicefrac{n}{2} \Big] = 1.
    % \]

    Take \(\bm{p} = \big(p_1, \dots, p_n\big)\) to be the vector of competences of \(n\) voters.\\

    Note, first, that if we improve the competence of one voter, 
    then the probability of a correct majority decision increases.\\

    Formally, suppose we replace some \(p_i\) in \(\bm{p}\) with \(p'_i > p_i\), 
    while keeping all other competences the same.
    We say the resulting vector \(\bm{p}'\) is \emph{improvement} of \(\bm{p}\).\\
    
    If \(S'_n\) is the sum of the votes determined by \(\bm{p}'\), we have that: 
    \[
      {\Pr}\Big[ S'_n > \nicefrac{n}{2}\Big] > {\Pr}\Big[ S_n > \nicefrac{n}{2}\Big]
    \]

    Note, now, that we can get from \(\bm{p}^* = \Big(\nicefrac{1}{2} + \epsilon, \dots, \nicefrac{1}{2} + \epsilon \Big)\)
    to any \(\bm{p} = \Big(p_1, \dots, p_n\Big)\) by a series of improvements.\\

    But we already know, from the Condorcet Jury Theorem, that the group accuracy of \(\bm{p}^*\) approaches \(1\) asymptotically.\\

    So the accuracy of \(\bm{p}\) does the same.\\

    % The expected value of the random variable \(v_i\) is:
    % \begin{flalign*}
    %     \qquad\EXP\big[v_i\big] & = 1 \cdot p_i + 0 \cdot (1-p) &\\ 
    %               & = p_i.&
    % \end{flalign*}
    % This means that:
    % \begin{flalign*}
    %     \qquad\EXP\big[S_n\big] & = \EXP\big[v_1\big] + \EXP\big[v_2\big] + \dots + \EXP\big[v_n\big] &\\
    %                & = p_1 + p_2 + \dots + p_n.&
    % \end{flalign*}







    %% Model for cascades
    % \begin{table}
    %     \begin{tabular}{rl}
    %         agents & \(N = \{1, \dots , n\}\) \\
    %         alternatives & \(A = \{a,b\}\) \\
    %         better alternative & \(\theta \in A\), we usually assume \(\theta = a\) \\
    %         voter \(i\)'s signal & \(s_i \in A\)\\
    %         probability of a correct signal \(i\)'s & \(\Pr[s_i = \theta ] = p\), with \(p > \nicefrac{1}{2}\) \\
    %         agent \(i\)'s opinion & \(v_i \in A\) \\
    %                            & agents speak out in sequence, and see previous opinions
    %     \end{tabular}
    % \end{table}

    % \begin{table}
    %     \begin{tabular}{rccc}
    %         \toprule
    %               & pros & cons & net \\
    %         \midrule
    %         \(a\) & \(a_1, a_2, a_3, a_4\) & \(\overline{a}_1\) & 3 \\
    %         \(b\) & \(b_1, b_2\) & \(\emptyset\) & 2 \\
    %         \bottomrule
    %     \end{tabular}
    % \end{table}

    % \begin{table}
    %     \begin{tabular}{rccc}
    %         \toprule
    %               & pros & cons & net \\
    %         \midrule
    %         \(a\) & \(a_3, a_4\) & \(\overline{a}_1\) & 1 \\
    %         \(b\) & \(b_1,b_2\) & \(\emptyset\) & 2 \\
    %         \bottomrule
    %     \end{tabular}
    % \end{table}

    % \(b_1, b_2, \overline{a}_1\)
\end{document}