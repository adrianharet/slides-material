%% magick convert -density 1200 test.pdf test.png
\documentclass[preview, border={0pt 1pt 0pt 1pt}, varwidth=10cm]{standalone} % border options are {left bottom right top}

\usepackage[T1]{fontenc}
% \usepackage[sfdefault]{AlegreyaSans}
% \renewcommand*\oldstylenums[1]{{\AlegreyaSansOsF #1}}

\usepackage[sfdefault]{FiraSans} %% option 'sfdefault' activates Fira Sans as the default text font
\usepackage[T1]{fontenc}
\renewcommand*\oldstylenums[1]{{\firaoldstyle #1}}

% \usepackage[T1]{fontenc}
% \usepackage{Alegreya} %% Option 'black' gives heavier bold face 
% \renewcommand*\oldstylenums[1]{{\AlegreyaOsF #1}}


% \usepackage[default]{lato}
% \usepackage[T1]{fontenc}

% \usepackage[sfdefault]{cabin}
% \usepackage[T1]{fontenc}

% \usepackage[sfdefault]{roboto}  %% Option 'sfdefault' only if the base font of the document is to be sans serif
% \usepackage[T1]{fontenc}

% \usepackage[sfdefault]{noto}
% \usepackage[T1]{fontenc}

% \usepackage[T1]{fontenc}
% \usepackage[sfdefault]{josefin}

\usepackage{amsthm}
\usepackage{amssymb}
\usepackage{amsmath}
\usepackage{bm}
\usepackage{nicefrac}
\usepackage{graphicx}
\usepackage{tikz}
\usepackage{booktabs}
\usepackage{soul}

\usetikzlibrary{automata, positioning}

%% Colors
% grey
\definecolor{DarkCharcoal}{RGB}{30, 30, 30} % theorem
\definecolor{Davys}{RGB}{60, 60, 60} % proof
\definecolor{SonicSilver}{RGB}{90, 90, 90} % proposition
\definecolor{Gray}{RGB}{120, 120, 120} % lemma
\definecolor{Spanish}{RGB}{150, 150, 150} % corollary
\definecolor{Argent}{RGB}{180, 180, 180} % example

% blues
\definecolor{Azure}{RGB}{0, 128, 255}
\definecolor{Egyptian}{RGB}{16, 52, 166}
\definecolor{Navy}{RGB}{17, 30, 108}
\definecolor{CeruleanFrost}{RGB}{104, 145, 195}
\definecolor{LightSteelBlue}{RGB}{170, 197, 226}

% reds
\definecolor{Renate}{RGB}{227, 25, 113}
\definecolor{Burgundy}{RGB}{141, 2, 31}
\definecolor{Salmon}{RGB}{250, 128, 114}

% yellows
\definecolor{Honey}{RGB}{235, 150, 5}

% greens
\definecolor{PastelGreen}{HTML}{6ECB63}
\definecolor{FernGreen}{HTML}{116530}
\definecolor{CadmiumGreen}{HTML}{046C41}


%% Color Schemes
% Pastel Tones Color Scheme
\definecolor{Manatee}{RGB}{154, 145, 172} % darkest
\definecolor{Lilac}{RGB}{202, 167, 189}
\definecolor{CherryBlossomPink}{RGB}{255, 185, 196}
\definecolor{LightRed}{RGB}{255, 211, 212}

% Luxury Gradient Color Scheme
\definecolor{PhilippineBronze}{RGB}{112, 54, 14}
\definecolor{GoldenBrown}{RGB}{147, 93, 35}
\definecolor{UniversityOfCaliforniaGold}{RGB}{182, 131, 55}
\definecolor{Sunray}{RGB}{217, 170, 76}
\definecolor{OrangeYellowCrayola}{RGB}{252, 208, 96}

% Breaking Apart Color Scheme
\definecolor{DeepTuscanRed}{RGB}{103, 74, 82}
\definecolor{FuzzyWuzzy}{RGB}{199, 114, 113}
\definecolor{MiddleYellowRed}{RGB}{232, 187, 109}
\definecolor{Independence}{RGB}{83, 70, 103}
\definecolor{CeladonGreen}{RGB}{47, 129, 133}

% Lilac
\definecolor{Lilac1}{HTML}{97cded} % blue
\definecolor{Lilac2}{HTML}{02a0da}
\definecolor{Lilac3}{HTML}{b8afc9} % purple
\definecolor{Lilac4}{HTML}{eae1ef}
\definecolor{Lilac5}{HTML}{eadbd7} % light coffee

% Monochrome Coffee
\definecolor{MonoCoffee1}{HTML}{281E15} % dark
\definecolor{MonoCoffee2}{HTML}{47423E}
\definecolor{MonoCoffee3}{HTML}{928477}
\definecolor{MonoCoffee4}{HTML}{B7B2AC}
\definecolor{MonoCoffee5}{HTML}{A49FA3} % light

% Monochrome Amethyst
\definecolor{MonoAmethyst1}{HTML}{22222E} % dark
\definecolor{MonoAmethyst2}{HTML}{393A5A}
\definecolor{MonoAmethyst3}{HTML}{706F8E}
\definecolor{MonoAmethyst4}{HTML}{ADA9BA}
\definecolor{MonoAmethyst5}{HTML}{E9E9E9} % light

% Warm Brown
\definecolor{WarmBrown1}{HTML}{3C0906} % dark
\definecolor{WarmBrown2}{HTML}{84352D}
\definecolor{WarmBrown3}{HTML}{BC5F41}
\definecolor{WarmBrown4}{HTML}{E4E4E6}
\definecolor{WarmBrown5}{HTML}{E4C08A} % light

% FMAS course
\definecolor{GameTheory}{HTML}{AA2B1D}
\definecolor{Voting}{HTML}{FE9801}
\definecolor{Matching}{HTML}{F9B384}
\definecolor{Auctions}{HTML}{583D72}
\definecolor{ABMs}{HTML}{374045}
\definecolor{WisdomOfCrowds}{HTML}{7189BF}

% environments
\definecolor{colorTheorem}{HTML}{DA2D2D}
\definecolor{colorProposition}{HTML}{FF5733}
\definecolor{colorProof}{HTML}{621055}
\definecolor{colorDefinition}{HTML}{FF9F45}
\definecolor{colorAxiom}{HTML}{506D84}
\definecolor{colorProcedure}{HTML}{753422}
\definecolor{colorProblem}{HTML}{374045}
\definecolor{colorObservation}{HTML}{EA9ABB}
\definecolor{colorABM}{HTML}{89B5AF}
\definecolor{colorExample}{HTML}{97C4B8}
\definecolor{colorConjecture}{RGB}{217, 170, 76}


\newcommand{\EXP}{\mathbb{E}}
\newcommand{\maj}{{\textit{maj}}}
\renewcommand{\epsilon}{\varepsilon}


\begin{document}
    % Compartment models
    % We have \(N\) agents in an environment.\\

    % At any time-step \(t\), agents are in one of several possible \emph{states}.\\
    
    % \emph{Infected} agents have caught the infection.
    % The number of infected agents at time \(t\) is \(I_t\).\\

    % \emph{Susceptible} agents are not infected, but they can become infected.
    % The number of susceptible agents at time \(t\) is \(S_t\).\\

    % \emph{Removed}, or \emph{recovered} agents are not infected and they 
    % cannot become infected, either because they are immune, or because they 
    % are dead. The number of removed agents at time \(t\) is \(R_t\).

    % Aims
    % We're interested in the dynamics of \(I_t\), \(S_t\) and \(R_t\) over time.

    % The SA Model
    % Only two possible states: susceptible and infected.\\  

    % At time \(t\), a susceptible agent becomes infected 
    % with probability \(\alpha\).\\

    % Movement and social structure play no role.

    % The number of susceptibles at time \(t\) are the non-infected agents:
    % \begin{flalign*}
    %     \qquad S_t = N - I_t. &&
    % \end{flalign*}
    % The average number of newly infected at time \(t\) is \(\alpha S_t\), or: 
    % \begin{flalign*}
    %     \qquad \alpha (N-I_t).&&
    % \end{flalign*}
    % Thus, the number of infected at time \(t+1\) is given by the recursion:
    % \begin{flalign*}
    %     \qquad I_{t+1} &= I_t + \alpha (N - I_t), &&
    % \end{flalign*}
    % which, written as a \emph{difference equation}, gives: 
    % \begin{flalign*}
    %     \qquad \Delta I & = I_{t+1} - I_t &&\\ 
    %              & = \alpha (N - I_t). &&
    % \end{flalign*}

    % When we plot the proportion of infected agents over time, we do not see 
    % the S-shape we are looking for.

    % The SI model
    % Only two possible states: susceptible and infected.\\  

    % At time \(t\) agents form pairs.\\

    % This can be approximated by agents moving around and getting close to each other.
    % Social structure now plays a role!\\

    % An infected agent transmits the disease to a nearby susceptible agent with probability \(\tau\).

    % Imagine a random variable \(X_i\) that keeps track of whether agent \(i\)
    % gets infected at time \(t\):
    % \begin{flalign*}
    %     \qquad X_i & = 
    %     \begin{cases}
    %         1,\text{if $i$ gets infected}, \\
    %         0,\text{otherwise}.    
    %     \end{cases} &&
    % \end{flalign*}
    % The probability of agent \(i\) getting infected at time \(t\), assuming that \(i\) bumps 
    % into another agent \(j\), is:
    % \begin{flalign*}
    %     \qquad \Pr[X_i = 1] & = {\Pr}\big[\text{\(i\) is susceptible, \(j\) is infected, \(j\) passes on the infection}\big]&&\\
    %                         & = \Pr[\text{\(i\) is susceptible}] \cdot \Pr[\text{\(j\) is infected}] \cdot \Pr[\text{\(j\) passes on the infection}]&&\\
    %                         & = \frac{N-I_t}{N} \cdot \frac{I_t}{N} \cdot \tau.
    % \end{flalign*}
    % The average number of agents becoming infected at time \(t\) is, then:
    % \begin{flalign*}
    %     \qquad \EXP\left[\sum_{1}^{N}X_i\right] & = \EXP[X_1] + \dots + \EXP[X_N] &&\\
    %                                             & = N \cdot \frac{N-I_t}{N} \cdot \frac{I_t}{N} \cdot \tau && \\
    %                                             & = \tau I_t\left(1-\frac{I_t}{N}\right).
    % \end{flalign*}

    % Hence, the recursion relation for the number of infectious agents at time \(t+1\) is:
    % \begin{flalign*}
    %     \qquad I_{t+1} &= I_t + \tau I_t\left(1-\frac{I_t}{N}\right).&&
    % \end{flalign*}

    % We see the desired S-shape. Social influence has done the trick!

    % The SIS model
    % Only two possible states: susceptible and infected.\\  

    % A proportion \(V\) of the population starts out vaccinated, which means they are immune.\\

    % At time \(t\) agents form pairs, as if from getting close to each other.\\

    % An infected agent transmits the disease to a nearby susceptible agent with probability \(\tau\).\\

    % An infected agent becomes susceptible again with probability \(\gamma\).


    % Apart from the susceptibles that catch the infection, an average of: 
    % \begin{flalign*}
    %     \qquad \gamma I_t &&
    % \end{flalign*}
    % infected agents become susceptible again at time \(t\).\\

    % Thus, the recurrence relation becomes:
    % \begin{flalign*}
    %     \qquad I_{t+1} &= I_t + \tau I_t\left(1-\frac{I_t}{N}\right) - \gamma I_t.&&
    % \end{flalign*}

    % A \emph{dynamic equilibrium} occurs when the number of infected agents stabilizes. 

    % The dynamic equilibrium is obtained by setting \(I_{t+1} = I_t = I\), and plugging this
    % into the recurrence relation to get:
    % \begin{flalign*}
    %     \qquad I = I + \tau I\left(1-\frac{I}{N}\right) - \gamma I \quad& \text{iff} \quad \tau\left(1-\frac{I}{N}\right) = \gamma & \\
    %                                                                & \text{iff}\quad \frac{I}{N} = 1 - \frac{\gamma}{\tau}. &
    % \end{flalign*}

    % At the beginning of an infection, the number \(I_t\) of infected agents is close to \(0\),
    % hence:
    % \begin{flalign*}
    %     \qquad 1- \frac{I_t}{N} \approx 1.&&
    % \end{flalign*}
    % Plugging this into the recurrence relation, we have:
    % \begin{flalign*}
    %     \qquad I_{t+1} & = I_t + \tau I_t\left(1-\frac{I_t}{N}\right) - \gamma I_t.&&\\
    %                    & \approx I_t + \tau \cdot I_t - \gamma \cdot I_t &&\\
    %                    & = I_t + (\tau - \gamma) I_t.
    % \end{flalign*}
    % The condition for the infection spreading becomes:
    % \begin{flalign*}
    %     \qquad \tau - \gamma > 0 \quad & \text{iff}\quad \frac{\tau}{\gamma} > 1.&
    % \end{flalign*}

    % The \emph{basic reproduction number} is defined as: 
    % \begin{flalign*}
    %     \qquad R_0 = \frac{\tau}{\gamma}. &&
    % \end{flalign*}
    % We have just shown that, under the assumptions of the basic SIS model,
    % infection spreads just in case \(R_0 >1\).

    % Recall how we keep track of infected agents:
    % \begin{flalign*}
    %     \qquad X_i & = 
    %     \begin{cases}
    %         1,\text{if $i$ gets infected}, \\
    %         0,\text{otherwise}.    
    %     \end{cases} &&
    % \end{flalign*}
    % With vaccination, the probability of agent \(i\) getting infected by \(j\) at time \(t\) 
    % depends on \(i\) being unvaccinated:
    % \begin{flalign*}
    %     \qquad \Pr[X_i = 1] & = {\Pr}\big[\text{\(i\) is susceptible, \(i\) is unvaccinated, \(j\) is infected,} &&\\
    %                         & \qquad\qquad\text{\(j\) passes on the infection}\big]&&\\
    %                         & = \frac{N-I_t}{N} \cdot (1-V) \cdot \frac{I_t}{N} \cdot \tau.
    % \end{flalign*}
    % Thus, the average number of newly infected agents at \(t\) is:
    % \begin{flalign*}
    %     \qquad \tau\left(1 - \frac{I_t}{N}\right)(1-V)I_t &&
    % \end{flalign*}

    % With vaccinated agents, the recurrence relation for the change in infected agents is:
    % \begin{flalign*}
    %     \qquad I_{t+1} &= I_t + \tau\left(1-\frac{I_t}{N}\right)(1-V)I_t - \gamma I_t.&&
    % \end{flalign*}
    % Approximating \(1-\nicefrac{I_t}{N}\) with \(1\) again, the condition for the infection spreading is:
    % \begin{flalign*}
    %     \qquad \tau(1-V) - \gamma > 0 \quad & \text{iff}\quad \frac{\tau}{\gamma}(1-V) > 1 &\\
    %                                         & \text{iff}\quad R_0(1-V) > 1.
    % \end{flalign*} 

    % The \emph{effective basic reproductive number} is:
    % \begin{flalign*}
    %     \qquad r_0 & = R_0(1-V) &&\\
    %                & = \frac{\tau}{\gamma}(1 - V).
    % \end{flalign*}

    % We have just shown that the infection spreads just in case \(r_0 > 1\).

    % The infection does \emph{not} spread just in case:
    % \begin{flalign*}
    %     \qquad R_0(1-V) \leq 1 \quad & \text{iff}\quad 1-V \leq \frac{1}{R_0} & \\
    %                                  & \text{iff}\quad V \geq 1 - \frac{1}{R_0}.
    % \end{flalign*}
    % The smallest value for which the infection does not spread, called the \emph{threshold vaccination rate for herd immunity}, 
    % is:
    % \begin{flalign*}
    %     \qquad V^* = 1-\frac{1}{R_0}. &&
    % \end{flalign*}

    % The SIR Model
    Three possible states: susceptible, infected and removed\\  

    At time \(t\) agents form pairs, as if from getting close to each other.\\

    An infected agent transmits the disease to a nearby susceptible agent with probability \(\tau\).\\

    An infected agent becomes removed with probability \(\tau\).

    % The dynamics are given by the following recurrence relations:
    % \begin{flalign*}
    %     \qquad S_{t+1} & = S_t - \tau S_t \frac{I_t}{N}, &&\\
    %     \qquad I_{t+1} & = I_t + \tau S_t \frac{I_t}{N} - \gamma I_t, &&\\
    %     \qquad R_{t+1} & = R_t + \gamma I_t. &&
    % \end{flalign*}
    % Written as difference equations:
    % \begin{flalign*}
    %     \qquad \Delta S & = -\tau S_t \frac{I}{N}, &&\\
    %     \qquad \Delta I & = \tau S \frac{I}{N} - \gamma I_t, &&\\
    %     \qquad \Delta R & = \gamma I. &&
    % \end{flalign*}
\end{document}