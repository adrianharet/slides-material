%% magick convert -density 1200 test.pdf test.png
\documentclass[
    preview, 
    varwidth = 8cm, 
    border = {2pt 0pt 0pt 0pt}
    ]{standalone} % border options are {left bottom right top}

\usepackage[T1]{fontenc}
% \usepackage[sfdefault]{AlegreyaSans}
% \renewcommand*\oldstylenums[1]{{\AlegreyaSansOsF #1}}

\usepackage[sfdefault]{FiraSans} %% option 'sfdefault' activates Fira Sans as the default text font
\usepackage[T1]{fontenc}
\renewcommand*\oldstylenums[1]{{\firaoldstyle #1}}

% \usepackage[T1]{fontenc}
% \usepackage{Alegreya} %% Option 'black' gives heavier bold face 
% \renewcommand*\oldstylenums[1]{{\AlegreyaOsF #1}}


% \usepackage[default]{lato}
% \usepackage[T1]{fontenc}

% \usepackage[sfdefault]{cabin}
% \usepackage[T1]{fontenc}

% \usepackage[sfdefault]{roboto}  %% Option 'sfdefault' only if the base font of the document is to be sans serif
% \usepackage[T1]{fontenc}

% \usepackage[sfdefault]{noto}
% \usepackage[T1]{fontenc}

% \usepackage[T1]{fontenc}
% \usepackage[sfdefault]{josefin}

\usepackage{amsthm}
\usepackage{amssymb}
\usepackage{amsmath}
\usepackage{bm}
\usepackage{nicefrac}
\usepackage{graphicx}
\usepackage{tikz}
\usepackage{booktabs}
\usepackage{soul}

\usetikzlibrary{automata, positioning}

%% Colors
% grey
\definecolor{DarkCharcoal}{RGB}{30, 30, 30} % theorem
\definecolor{Davys}{RGB}{60, 60, 60} % proof
\definecolor{SonicSilver}{RGB}{90, 90, 90} % proposition
\definecolor{Gray}{RGB}{120, 120, 120} % lemma
\definecolor{Spanish}{RGB}{150, 150, 150} % corollary
\definecolor{Argent}{RGB}{180, 180, 180} % example

% blues
\definecolor{Azure}{RGB}{0, 128, 255}
\definecolor{Egyptian}{RGB}{16, 52, 166}
\definecolor{Navy}{RGB}{17, 30, 108}
\definecolor{CeruleanFrost}{RGB}{104, 145, 195}
\definecolor{LightSteelBlue}{RGB}{170, 197, 226}

% reds
\definecolor{Renate}{RGB}{227, 25, 113}
\definecolor{Burgundy}{RGB}{141, 2, 31}
\definecolor{Salmon}{RGB}{250, 128, 114}

% yellows
\definecolor{Honey}{RGB}{235, 150, 5}

% greens
\definecolor{PastelGreen}{HTML}{6ECB63}
\definecolor{FernGreen}{HTML}{116530}
\definecolor{CadmiumGreen}{HTML}{046C41}


%% Color Schemes
% Pastel Tones Color Scheme
\definecolor{Manatee}{RGB}{154, 145, 172} % darkest
\definecolor{Lilac}{RGB}{202, 167, 189}
\definecolor{CherryBlossomPink}{RGB}{255, 185, 196}
\definecolor{LightRed}{RGB}{255, 211, 212}

% Luxury Gradient Color Scheme
\definecolor{PhilippineBronze}{RGB}{112, 54, 14}
\definecolor{GoldenBrown}{RGB}{147, 93, 35}
\definecolor{UniversityOfCaliforniaGold}{RGB}{182, 131, 55}
\definecolor{Sunray}{RGB}{217, 170, 76}
\definecolor{OrangeYellowCrayola}{RGB}{252, 208, 96}

% Breaking Apart Color Scheme
\definecolor{DeepTuscanRed}{RGB}{103, 74, 82}
\definecolor{FuzzyWuzzy}{RGB}{199, 114, 113}
\definecolor{MiddleYellowRed}{RGB}{232, 187, 109}
\definecolor{Independence}{RGB}{83, 70, 103}
\definecolor{CeladonGreen}{RGB}{47, 129, 133}

% Lilac
\definecolor{Lilac1}{HTML}{97cded} % blue
\definecolor{Lilac2}{HTML}{02a0da}
\definecolor{Lilac3}{HTML}{b8afc9} % purple
\definecolor{Lilac4}{HTML}{eae1ef}
\definecolor{Lilac5}{HTML}{eadbd7} % light coffee

% Monochrome Coffee
\definecolor{MonoCoffee1}{HTML}{281E15} % dark
\definecolor{MonoCoffee2}{HTML}{47423E}
\definecolor{MonoCoffee3}{HTML}{928477}
\definecolor{MonoCoffee4}{HTML}{B7B2AC}
\definecolor{MonoCoffee5}{HTML}{A49FA3} % light

% Monochrome Amethyst
\definecolor{MonoAmethyst1}{HTML}{22222E} % dark
\definecolor{MonoAmethyst2}{HTML}{393A5A}
\definecolor{MonoAmethyst3}{HTML}{706F8E}
\definecolor{MonoAmethyst4}{HTML}{ADA9BA}
\definecolor{MonoAmethyst5}{HTML}{E9E9E9} % light

% Warm Brown
\definecolor{WarmBrown1}{HTML}{3C0906} % dark
\definecolor{WarmBrown2}{HTML}{84352D}
\definecolor{WarmBrown3}{HTML}{BC5F41}
\definecolor{WarmBrown4}{HTML}{E4E4E6}
\definecolor{WarmBrown5}{HTML}{E4C08A} % light

% FMAS course
\definecolor{GameTheory}{HTML}{AA2B1D}
\definecolor{Voting}{HTML}{FE9801}
\definecolor{Matching}{HTML}{F9B384}
\definecolor{Auctions}{HTML}{583D72}
\definecolor{ABMs}{HTML}{374045}
\definecolor{WisdomOfCrowds}{HTML}{7189BF}

% environments
\definecolor{colorTheorem}{HTML}{DA2D2D}
\definecolor{colorProposition}{HTML}{FF5733}
\definecolor{colorProof}{HTML}{621055}
\definecolor{colorDefinition}{HTML}{FF9F45}
\definecolor{colorAxiom}{HTML}{506D84}
\definecolor{colorProcedure}{HTML}{753422}
\definecolor{colorProblem}{HTML}{374045}
\definecolor{colorObservation}{HTML}{EA9ABB}
\definecolor{colorABM}{HTML}{89B5AF}
\definecolor{colorExample}{HTML}{97C4B8}
\definecolor{colorConjecture}{RGB}{217, 170, 76}


\newcommand{\EXP}{\mathbb{E}}
\newcommand{\maj}{{\textit{maj}}}
\renewcommand{\epsilon}{\varepsilon}



\begin{document}\raggedright
% We represent a \emph{social network} as a graph $G = (V, E)$, 
% where $V$ is the set of \emph{vertices} (agents) and 
% $E$ is the set of \emph{edges} (relationships).
% \vspace{1em}

% The graph can be \emph{directed} or \emph{undirected}, 
% depending on whether relationships are one-way or two-way.


% The \emph{degree} \(d_i\) of a node \(i\) is the number of edges connected to it:
% \begin{align*}
%     d_i & = \big|\big\{j \in V \mid (i, j) \in E\big\}\big|.&
% \end{align*}





% The \emph{degree distribution} \(P(d)\) of a network is a description of the relative frequencies of nodes that
% have different degrees. 
% \vspace{1em}

% In other words, \(P(d)\) is the fraction of nodes that have degree \(d\).
% \vspace{1em}

% The function \(P\) can be a frequency distribution, if we are describing a specific network;
% or a probability distribution, if we are working with random networks.



% For a set of \(n\) nodes and a probability \(p\), the \emph{Erdős-Rényi model} generates a random graph \(G(n, p)\) 
% by going through all possible pairs of nodes and adding an edge between them with probability \(p\).




% The degree distribution of an Erdős-Rényi random graph \(G(n, p)\) is given by the \emph{binomial distribution}:
% \[
%     {\Pr}\Big[ d = k \Big] = {n-1 \choose k} p^k (1-p)^{n-1-k}.
% \]

% The `typical' node has degree \((n-1)p\).



% A \emph{scale-free network} is a type of network whose degree distribution follows a power law:
% \[
%     P(d) \sim c \cdot d^{-\gamma},
% \]
% where \(c>0\) and \(\gamma\) is a constant typically in the range \(2 < \gamma < 3\).




% The relative frequencies stay constant as the degree grows from \(d\) to \(kd\):
% \begin{flalign*}
%     \qquad
%     \frac{P(kd)}{P(d)} & = \frac{c \cdot k^{-\gamma} d ^{-\gamma}}{c \cdot d^{-\gamma}} &\\ 
%                        & = k^{-\gamma}. &
% \end{flalign*}
% This is true regardless of the degree \(d\) we start with:
% \begin{flalign*}
%     \qquad
%     \frac{P(2)}{P(1)} & = \frac{P(20)}{P(10)} = \frac{P(200)}{P(100)} = \dots&
% \end{flalign*}
% Best seen when graphing the degree distribution\dots
% \vspace{1em}

% \dots on a \emph{log-log scale}, for readability:
% the degree distribution of a scale-free network appears as a straight line with slope \(-\gamma\).


% Start with \(m_0\) nodes. New nodes are added, one at a time.
% Each new node connects to an existing node \(i\) with a probability \(p_i\) 
% proportional to \(i\)'s degree:
% \begin{flalign*}
%     \qquad
%     p_i & = \frac{d_i}{\sum_{j} d_j}. &
% \end{flalign*}







% The \emph{distance} \(d(i, j)\) between two nodes \(i\) and \(j\) 
% in a network is the length of the shortest path between \(i\) and \(j\).

% The \emph{diameter} of a network \(G\) is the maximum distance between any two nodes:
% \begin{flalign*}
%     \qquad
%     \text{diam}(G) & = \max_{i, j \in V} d(i, j). &
% \end{flalign*}


% The maximum distance (i.e., diameter) is 
% \(2\) in \(G_1\). 
% \vspace{1em}

% And \(4\) in \(G_2\).


% The \emph{neighborhood} \(N(i)\) of a node \(i\) is the set of nodes that are directly connected to \(i\):
% \begin{flalign*}
%     \qquad
%     N(i) & = \big\{j \in V \mid (i, j) \in E\big\}. &
% \end{flalign*}


% The \emph{degree} \(d_i\) of a node \(i\) is the size of \(i\)'s neighborhood:
% \begin{flalign*}
%     \qquad
%     d_i & = \big|N(i) \big|, &
% \end{flalign*}
% i.e., the number of nodes directly connected to \(i\).


% Consider the \emph{star graph} \(G_1\).
% \vspace{1em}

% And the \emph{(binary) tree} \(G_2\).
% \vspace{1em}

% The distance between \(i\) and \(j\) is \(2\) in \(G_1\) and \(4\) in \(G_2\).


% The \emph{average shortest path length} \(\mathit{L}(G)\) of a network \(G\) is the average distance between all pairs of nodes:
% \begin{flalign*}
%     \qquad
%     \mathit{L}(G) & = \frac{1}{|V|(|V|-1)} \sum_{i, j \in V} d(i, j). &
% \end{flalign*}


% The \emph{clustering coefficient} \(\mathit{Cl}(i)\) of a node \(i\) is the fraction of pairs of its neighbors 
% connected to each other:
% \begin{flalign*}
%     \quad
%     \mathit{Cl}(i) & = \frac{\text{number of edges between neighbors of } i}{\text{number of pairs of neighbors of } i} 
%                      = \frac{\bigg|\Big\{\{j, k\} \subseteq N(i) \mid (j, k) \in E\Big\}\bigg|}{\bigg|\Big\{\{j, k\} \subseteq N(i)\Big\}\bigg|} & \\
%                    & = \frac{\bigg|\Big\{\{j, k\} \subseteq N(i) \mid (j, k) \in E\Big\}\bigg|}{\binom{d_i}{2}}  &\\
%                      & = 2 \cdot \frac{\bigg|\Big\{\{j, k\} \subseteq N(i) \mid (j, k) \in E\Big\}\bigg|}{d_i(d_i - 1)}. &
% \end{flalign*}


% The \emph{clustering coefficient} \(\mathit{Cl}(G)\) of network \(G\) is the average clustering coefficient of its nodes:
% \begin{flalign*}
%     \qquad
%     \mathit{Cl}(G) & = \frac{1}{|V|} \sum_{i \in V} \mathit{Cl}(i). &
% \end{flalign*}


% A \emph{small-world network} is a type of network characterized by a \emph{high clustering coefficient} and 
% \emph{short average path length}:
%     \[
%         \mathit{L}(G) \sim \log(|V|),
%     \]
% i.e., the average distance between any two nodes is proportional to the logarithm of the number of nodes.


% Start with a graph where each node is connected to exactly \(k\) other nodes. 
% Then, at every time step, randomly rewire each edge with probability \(p\) to an unattached node.



% The \emph{degree centrality}\(^*\) \(C_d(i)\) of node \(i\) is the number of nodes connected to \(i\):
% \begin{flalign*}
%     \qquad
%     C_d(i) & = d_i. &
% \end{flalign*}

% To make it easier to compare centralities across different networks, 
% we can normalize it relative to the total number \(n\) of nodes:
% \begin{flalign*}
%     \qquad
%     C_d(i) & = \frac{d_i}{n - 1}. &
% \end{flalign*}

% \(^*\)Note that this definition applies to undirected networks.




% Take \(\sigma_{j, k}\) to be \emph{the number of shortest paths between \(j\) and \(k\)}, 
% and \(\sigma_{j, k}(i)\) to be the number of shortest paths between \(j\) and \(k\) 
% \emph{that pass through \(i\)}.
% \vspace{1em}

% The \emph{betweeness centrality} of node \(i\) is defined as: 
% \begin{flalign*}
%     \qquad
%     C_b(i) & = \frac{1}{\nicefrac{(n-1)(n-2)}{2}} \cdot 
%         \sum_{j \neq k, j \neq i, k \neq i } \frac{\sigma_{j, k}(i)}{\sigma_{j, k}}, &
% \end{flalign*}
% i.e., the average fraction of shortest paths that pass through \(i\).



% There are \(\nicefrac{1}{2} \cdot (7-1) \cdot (7-2) = 15\) pairs of nodes that do not include node \(3\):
% \begin{flalign*}
%     \qquad
%     & (0, 1), (0, 2), (0, 4), (0, 5), (0, 6), &\\
%     & (1, 2), (1, 4), (1, 5), (1, 6),         &\\
%     & (2, 4), (2, 5), (2, 6),                 &\\
%     & (4, 5), (4, 6),                         &\\
%     & (5, 6).    
% \end{flalign*}
% The shortest path between \(0\) and \(1\) is \((0,1)\).
% The shortest path between \(0\) and \(4\) is \((0, 2, 3, 4)\)\dots 
% \vspace{1em}

% Every pair of nodes has a unique shortest path,
% i.e., \(\sigma_{j, k} = 1\) for all \(j\neq 3\) and \(k \neq 3\).
% \vspace{1em}

% Node \(3\) is on the shortest path for \(9\) out of the \(15\) pairs of nodes.
% \vspace{1em}

% The \emph{betweeness centrality} of node \(3\) is thus:
% \begin{flalign*}
%     \qquad
%     C_b(3) & = \frac{1}{\nicefrac{(7-1)(7-2)}{2}} \cdot 9 &\\
%            & = 0.6. &
% \end{flalign*}



% The \emph{eigenvector centrality} \(C_e(i)\) of node \(i\) is defined as:
% \begin{flalign*}
%     \qquad
%     C_e(i) & = \frac{1}{\lambda} \sum_{j \in N(i)} C_e(j), &
% \end{flalign*}
% where \(\lambda\) is the largest eigenvalue of the adjacency matrix \(A\) of the network,
% and \(N(i)\) is the neighborhood of node \(i\).


% For an \(n \times n\) matrix 
% \(
%     \bm{M} = 
%     \begin{bmatrix}
%         m_{11} & \dots & m_{1n}  \\
%         \vdots & \ddots & \vdots  \\
%         m_{n1} & \dots & m_{nn}
%     \end{bmatrix},
% \)
% an \emph{eigenvector} 
% \(
%     \bm{v} = 
%     \begin{bmatrix}
%         v_1  \\
%         \vdots \\
%         v_{n}
%     \end{bmatrix}
%     \) 
% is a non-zero \(n \times 1\) vector such that:
% \begin{flalign*}
%     \qquad
%     & \bm{M}\bm{v} = \lambda \bm{v}, &
% \end{flalign*}
% where \(\lambda\) is a scalar called the \emph{eigenvalue} of \(\bm{M}\) corresponding to the eigenvector \(\bm{v}\).


% \[
%     \bm{A} = 
%     \begin{bmatrix}
%         0 & 1 & 1 & 1 & 1  \\
%         1 & 0 & 0 & 1 & 0  \\
%         1 & 0 & 0 & 1 & 0  \\
%         1 & 1 & 1 & 0 & 1  \\
%         1 & 0 & 0 & 1 & 0
%     \end{bmatrix}
% \]

The \emph{adjacency matrix} \(\bm{A}\) of a network with \(n\) nodes is an \(n\times n\) matrix 
where entry \(a_{ij}\) is \(1\) if there is an edge between nodes \(i\) and \(j\), and \(0\) otherwise.

\end{document}

