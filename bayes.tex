%% magick convert -density 1200 test.pdf test.png
\documentclass[preview, border={0pt 2pt 1pt 1pt}, varwidth=9cm]{standalone} % border options are {left bottom right top}

\usepackage[T1]{fontenc}
% \usepackage[sfdefault]{AlegreyaSans}
% \renewcommand*\oldstylenums[1]{{\AlegreyaSansOsF #1}}

\usepackage[sfdefault]{FiraSans} %% option 'sfdefault' activates Fira Sans as the default text font
\usepackage[T1]{fontenc}
\renewcommand*\oldstylenums[1]{{\firaoldstyle #1}}

% \usepackage[T1]{fontenc}
% \usepackage{Alegreya} %% Option 'black' gives heavier bold face 
% \renewcommand*\oldstylenums[1]{{\AlegreyaOsF #1}}


% \usepackage[default]{lato}
% \usepackage[T1]{fontenc}

% \usepackage[sfdefault]{cabin}
% \usepackage[T1]{fontenc}

% \usepackage[sfdefault]{roboto}  %% Option 'sfdefault' only if the base font of the document is to be sans serif
% \usepackage[T1]{fontenc}

% \usepackage[sfdefault]{noto}
% \usepackage[T1]{fontenc}

% \usepackage[T1]{fontenc}
% \usepackage[sfdefault]{josefin}

\usepackage{amsthm}
\usepackage{amssymb}
\usepackage{amsmath}
\usepackage{bm}
\usepackage{nicefrac}
\usepackage{graphicx}
\usepackage{tikz}
\usepackage{booktabs}
\usepackage{soul}

\usetikzlibrary{automata, positioning}

%% Colors
% grey
\definecolor{DarkCharcoal}{RGB}{30, 30, 30} % theorem
\definecolor{Davys}{RGB}{60, 60, 60} % proof
\definecolor{SonicSilver}{RGB}{90, 90, 90} % proposition
\definecolor{Gray}{RGB}{120, 120, 120} % lemma
\definecolor{Spanish}{RGB}{150, 150, 150} % corollary
\definecolor{Argent}{RGB}{180, 180, 180} % example

% blues
\definecolor{Azure}{RGB}{0, 128, 255}
\definecolor{Egyptian}{RGB}{16, 52, 166}
\definecolor{Navy}{RGB}{17, 30, 108}
\definecolor{CeruleanFrost}{RGB}{104, 145, 195}
\definecolor{LightSteelBlue}{RGB}{170, 197, 226}

% reds
\definecolor{Renate}{RGB}{227, 25, 113}
\definecolor{Burgundy}{RGB}{141, 2, 31}
\definecolor{Salmon}{RGB}{250, 128, 114}

% yellows
\definecolor{Honey}{RGB}{235, 150, 5}

% greens
\definecolor{PastelGreen}{HTML}{6ECB63}
\definecolor{FernGreen}{HTML}{116530}
\definecolor{CadmiumGreen}{HTML}{046C41}


%% Color Schemes
% Pastel Tones Color Scheme
\definecolor{Manatee}{RGB}{154, 145, 172} % darkest
\definecolor{Lilac}{RGB}{202, 167, 189}
\definecolor{CherryBlossomPink}{RGB}{255, 185, 196}
\definecolor{LightRed}{RGB}{255, 211, 212}

% Luxury Gradient Color Scheme
\definecolor{PhilippineBronze}{RGB}{112, 54, 14}
\definecolor{GoldenBrown}{RGB}{147, 93, 35}
\definecolor{UniversityOfCaliforniaGold}{RGB}{182, 131, 55}
\definecolor{Sunray}{RGB}{217, 170, 76}
\definecolor{OrangeYellowCrayola}{RGB}{252, 208, 96}

% Breaking Apart Color Scheme
\definecolor{DeepTuscanRed}{RGB}{103, 74, 82}
\definecolor{FuzzyWuzzy}{RGB}{199, 114, 113}
\definecolor{MiddleYellowRed}{RGB}{232, 187, 109}
\definecolor{Independence}{RGB}{83, 70, 103}
\definecolor{CeladonGreen}{RGB}{47, 129, 133}

% Lilac
\definecolor{Lilac1}{HTML}{97cded} % blue
\definecolor{Lilac2}{HTML}{02a0da}
\definecolor{Lilac3}{HTML}{b8afc9} % purple
\definecolor{Lilac4}{HTML}{eae1ef}
\definecolor{Lilac5}{HTML}{eadbd7} % light coffee

% Monochrome Coffee
\definecolor{MonoCoffee1}{HTML}{281E15} % dark
\definecolor{MonoCoffee2}{HTML}{47423E}
\definecolor{MonoCoffee3}{HTML}{928477}
\definecolor{MonoCoffee4}{HTML}{B7B2AC}
\definecolor{MonoCoffee5}{HTML}{A49FA3} % light

% Monochrome Amethyst
\definecolor{MonoAmethyst1}{HTML}{22222E} % dark
\definecolor{MonoAmethyst2}{HTML}{393A5A}
\definecolor{MonoAmethyst3}{HTML}{706F8E}
\definecolor{MonoAmethyst4}{HTML}{ADA9BA}
\definecolor{MonoAmethyst5}{HTML}{E9E9E9} % light

% Warm Brown
\definecolor{WarmBrown1}{HTML}{3C0906} % dark
\definecolor{WarmBrown2}{HTML}{84352D}
\definecolor{WarmBrown3}{HTML}{BC5F41}
\definecolor{WarmBrown4}{HTML}{E4E4E6}
\definecolor{WarmBrown5}{HTML}{E4C08A} % light

% FMAS course
\definecolor{GameTheory}{HTML}{AA2B1D}
\definecolor{Voting}{HTML}{FE9801}
\definecolor{Matching}{HTML}{F9B384}
\definecolor{Auctions}{HTML}{583D72}
\definecolor{ABMs}{HTML}{374045}
\definecolor{WisdomOfCrowds}{HTML}{7189BF}

% environments
\definecolor{colorTheorem}{HTML}{DA2D2D}
\definecolor{colorProposition}{HTML}{FF5733}
\definecolor{colorProof}{HTML}{621055}
\definecolor{colorDefinition}{HTML}{FF9F45}
\definecolor{colorAxiom}{HTML}{506D84}
\definecolor{colorProcedure}{HTML}{753422}
\definecolor{colorProblem}{HTML}{374045}
\definecolor{colorObservation}{HTML}{EA9ABB}
\definecolor{colorABM}{HTML}{89B5AF}
\definecolor{colorExample}{HTML}{97C4B8}
\definecolor{colorConjecture}{RGB}{217, 170, 76}


\newcommand{\EXP}{\mathbb{E}}
\newcommand{\maj}{{\textit{maj}}}
\renewcommand{\epsilon}{\varepsilon}


\begin{document}
    \raggedright
    % \[
    %     \Pr [H \mid E] = \frac{\Pr[E \mid H] \cdot \Pr[H]}{\Pr[E]}.
    % \]



    % Nature decides whether a disease occurs (\(d\)) with \emph{prior probability}, 
    % or \emph{base rate}, \(b\):
    % \[
    %     \Pr[d] = b, \qquad \Pr[\lnot d] = 1-b.
    % \]
    % A test is administered, which returns either a \emph{positive result} (+),
    % or a \emph{negative result} (-).
    % \newline
    % 
    % A \emph{false positive} occurs with probability \(\alpha\):
    % \[
    %     \Pr[+ \mid \lnot d] = \alpha, \qquad \Pr[- \mid \lnot d] = 1 - \alpha.
    % \]
    % A \emph{false negative} occcurs with probability \(\beta\):
    % \[
    %     \Pr[- \mid d] = \beta, \qquad \Pr[+ \mid d] = 1 - \beta.
    % \]



    % For the MaterniT21 test, the base rate of the disease is one in \(2{,}500\) pregnancies,
    % i.e., \(\Pr[d] = 0.0004\).
    % \newline

    % When the disease is present, the test identifies it with probability \(0.99\)
    % (the test makers were not lying about this), 
    % so the false negative rate is \(\Pr[- \mid d] = 0.01\).
    % \newline

    % The false positive rate is \(\Pr[+ \mid \lnot d] = 0.00022\). Quite low!
    % \newline

    % Now suppose you're one of the women who just got a positive result.
    % \newline

    % Is this because of a genetic disease, or a false positive?
    % \newline

    % What you really want to find out is the probability of the disease 
    % \emph{given} a positive test: 
    % \[
    %     \Pr[d \mid +]=\,\, ?.
    % \]
    % How do we get this?



    % We know that \(\Pr[d] = 0.0004\), \(\Pr[- \mid d] = 0.01\) and \(\Pr[+ \mid \lnot d] = 0.00022\).
    % \newline

    % By Bayes' rule, we have:
    % \begin{flalign*}
    %     \qquad \Pr[d \mid +] & = \frac{\Pr[+ \mid d] \cdot \Pr[d]}{\Pr[+]}& \\ 
    %                          & = \frac{\Pr[+ \mid d] \cdot \Pr[d]}{\Pr[+ \mid d]\cdot \Pr[d] + \Pr[+ \mid \lnot d] \cdot \Pr[\lnot d]}\\ 
    %                          & = \frac{0.99 \cdot 0.0004}{0.99 \cdot 0.0004 + 0.00022 \cdot 0.9996}\\ 
    %                          & = 0.643.
    % \end{flalign*}
    % So about a third of the positive results is a dud.



    Suppose we know the base rate, false positive rate and false positive rate.
    \newline

    Testing once, we get a signal \(s_1 \in \{+, -\}\).
    Updating, we obtain the posterior \(\Pr\left[ d \mid s_1 \right]\).
    \newline

    Testing again, we obtain a new (independent) signal \(s_2 \in \{+, -\}\).
    \newline

    We update again, with \(\Pr\left[ d \mid s_1 \right]\) as the new prior:
    \begin{flalign*}
        \qquad \Pr[d \mid s_2, s_1] & = \frac{\Pr\left[ d, s_2, s_1 \right]}{\Pr\left[s_2, s_1 \right]}& \\ 
                               & = \frac{\Pr\left[ s_2 \mid d, s_1 \right] \cdot \Pr\left[ d, s_1 \right]}{\Pr\left[ s_1, s_2 \right]}\\ 
                               & = \frac{\big(\Pr\left[ s_2 \mid d \right] \cdot \Pr\left[ s_1 \mid d \right]\big) \cdot \Pr\left[ d \right]}{\Pr\left[ s_1, s_2 \right]}
    \end{flalign*}


\end{document}