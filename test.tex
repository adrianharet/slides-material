%% magick convert -density 1200 test.pdf test.png
\documentclass[preview, border={0pt 5pt 34pt 1pt}]{standalone} % border options are {left bottom right top}

% \usepackage[T1]{fontenc}
% \usepackage[sfdefault]{AlegreyaSans}
% \renewcommand*\oldstylenums[1]{{\AlegreyaSansOsF #1}}

% \usepackage[sfdefault]{FiraSans} %% option 'sfdefault' activates Fira Sans as the default text font
% \usepackage[T1]{fontenc}
% \renewcommand*\oldstylenums[1]{{\firaoldstyle #1}}

\usepackage[T1]{fontenc}
\usepackage{Alegreya} %% Option 'black' gives heavier bold face 
\renewcommand*\oldstylenums[1]{{\AlegreyaOsF #1}}


% \usepackage[default]{lato}
% \usepackage[T1]{fontenc}

% \usepackage[sfdefault]{cabin}
% \usepackage[T1]{fontenc}

% \usepackage[sfdefault]{roboto}  %% Option 'sfdefault' only if the base font of the document is to be sans serif
% \usepackage[T1]{fontenc}

% \usepackage[sfdefault]{noto}
% \usepackage[T1]{fontenc}

% \usepackage[T1]{fontenc}
% \usepackage[sfdefault]{josefin}

\usepackage{amsthm}
\usepackage{amssymb}
\usepackage{amsmath}
\usepackage{bm}
\usepackage{nicefrac}
\usepackage{graphicx}
\usepackage{tikz}
\usepackage{booktabs}
\usepackage{soul}

\usetikzlibrary{automata, positioning}

%% Colors
% grey
\definecolor{DarkCharcoal}{RGB}{30, 30, 30} % theorem
\definecolor{Davys}{RGB}{60, 60, 60} % proof
\definecolor{SonicSilver}{RGB}{90, 90, 90} % proposition
\definecolor{Gray}{RGB}{120, 120, 120} % lemma
\definecolor{Spanish}{RGB}{150, 150, 150} % corollary
\definecolor{Argent}{RGB}{180, 180, 180} % example

% blues
\definecolor{Azure}{RGB}{0, 128, 255}
\definecolor{Egyptian}{RGB}{16, 52, 166}
\definecolor{Navy}{RGB}{17, 30, 108}
\definecolor{CeruleanFrost}{RGB}{104, 145, 195}
\definecolor{LightSteelBlue}{RGB}{170, 197, 226}

% reds
\definecolor{Renate}{RGB}{227, 25, 113}
\definecolor{Burgundy}{RGB}{141, 2, 31}
\definecolor{Salmon}{RGB}{250, 128, 114}

% yellows
\definecolor{Honey}{RGB}{235, 150, 5}

% greens
\definecolor{PastelGreen}{HTML}{6ECB63}
\definecolor{FernGreen}{HTML}{116530}
\definecolor{CadmiumGreen}{HTML}{046C41}


%% Color Schemes
% Pastel Tones Color Scheme
\definecolor{Manatee}{RGB}{154, 145, 172} % darkest
\definecolor{Lilac}{RGB}{202, 167, 189}
\definecolor{CherryBlossomPink}{RGB}{255, 185, 196}
\definecolor{LightRed}{RGB}{255, 211, 212}

% Luxury Gradient Color Scheme
\definecolor{PhilippineBronze}{RGB}{112, 54, 14}
\definecolor{GoldenBrown}{RGB}{147, 93, 35}
\definecolor{UniversityOfCaliforniaGold}{RGB}{182, 131, 55}
\definecolor{Sunray}{RGB}{217, 170, 76}
\definecolor{OrangeYellowCrayola}{RGB}{252, 208, 96}

% Breaking Apart Color Scheme
\definecolor{DeepTuscanRed}{RGB}{103, 74, 82}
\definecolor{FuzzyWuzzy}{RGB}{199, 114, 113}
\definecolor{MiddleYellowRed}{RGB}{232, 187, 109}
\definecolor{Independence}{RGB}{83, 70, 103}
\definecolor{CeladonGreen}{RGB}{47, 129, 133}

% Lilac
\definecolor{Lilac1}{HTML}{97cded} % blue
\definecolor{Lilac2}{HTML}{02a0da}
\definecolor{Lilac3}{HTML}{b8afc9} % purple
\definecolor{Lilac4}{HTML}{eae1ef}
\definecolor{Lilac5}{HTML}{eadbd7} % light coffee

% Monochrome Coffee
\definecolor{MonoCoffee1}{HTML}{281E15} % dark
\definecolor{MonoCoffee2}{HTML}{47423E}
\definecolor{MonoCoffee3}{HTML}{928477}
\definecolor{MonoCoffee4}{HTML}{B7B2AC}
\definecolor{MonoCoffee5}{HTML}{A49FA3} % light

% Monochrome Amethyst
\definecolor{MonoAmethyst1}{HTML}{22222E} % dark
\definecolor{MonoAmethyst2}{HTML}{393A5A}
\definecolor{MonoAmethyst3}{HTML}{706F8E}
\definecolor{MonoAmethyst4}{HTML}{ADA9BA}
\definecolor{MonoAmethyst5}{HTML}{E9E9E9} % light

% Warm Brown
\definecolor{WarmBrown1}{HTML}{3C0906} % dark
\definecolor{WarmBrown2}{HTML}{84352D}
\definecolor{WarmBrown3}{HTML}{BC5F41}
\definecolor{WarmBrown4}{HTML}{E4E4E6}
\definecolor{WarmBrown5}{HTML}{E4C08A} % light

% FMAS course
\definecolor{GameTheory}{HTML}{AA2B1D}
\definecolor{Voting}{HTML}{FE9801}
\definecolor{Matching}{HTML}{F9B384}
\definecolor{Auctions}{HTML}{583D72}
\definecolor{ABMs}{HTML}{374045}
\definecolor{WisdomOfCrowds}{HTML}{7189BF}

% environments
\definecolor{colorTheorem}{HTML}{DA2D2D}
\definecolor{colorProposition}{HTML}{FF5733}
\definecolor{colorProof}{HTML}{621055}
\definecolor{colorDefinition}{HTML}{FF9F45}
\definecolor{colorAxiom}{HTML}{506D84}
\definecolor{colorProcedure}{HTML}{753422}
\definecolor{colorProblem}{HTML}{374045}
\definecolor{colorObservation}{HTML}{EA9ABB}
\definecolor{colorABM}{HTML}{89B5AF}
\definecolor{colorExample}{HTML}{97C4B8}
\definecolor{colorConjecture}{RGB}{217, 170, 76}


\newcommand{\EXP}{\mathbb{E}}
\newcommand{\maj}{{\textit{maj}}}
\renewcommand{\epsilon}{\varepsilon}


\begin{document}
    %% GAME THEORY
    % Player \(i\)'s \emph{best response} to the other players' strategies 
    % \(\bm{s}_{-i} = (s_1, \dots, s_{i-1}, s_{i+1}, \dots, s_n)\)
    % is a strategy \(s^{*}_i\) such that \(u_i(s^{*}_i, \bm{s}_{-i}) \geq u_i(s^*_i, \bm{s}_{-i})\), 
    % for any strategy \(s_i\) of player \(i\).
    % 
    % A strategy profile \(\bm{s}^{*} = (s_1^{*}, \dots, s_n^{*})\) is a 
    % \emph{pure Nash equilibrium} if \(s_i^{*}\) is a best response to \(\bm{s}^*_{-i}\),
    % for every player \(i\).
    % \vspace{1em}
    % 
    % In other words, given strategy profile \(\bm{s}^*\), there is no player \(i\) and strategy \(s'_i\)
    % such that \(u_i(s'_i, \bm{s}^*_{-i}) > u_i(s^*_i, \bm{s}^*_{-i})\).
    % 
    % A strategy profile \(\bm{s}\) \emph{Pareto dominates} strategy profile \(\bm{s}'\) if:
    % \begin{itemize}
    %     \item[(i)] \(u_i(\bm{s}) \geq u_i(\bm{s}')\), for every agent \(i\), and 
    %     \item[(ii)] there exists an agent \(j\) such that \(u_j(\bm{s}) > u_j(\bm{s}')\). 
    % \end{itemize}
    % 
    % A strategy profile \(\bm{s}\) is \emph{Pareto optimal} if 
    % there is no (other) strategy profile \(\bm{s}'\) that Pareto dominates \(\bm{s}\).
    % 
    % 
    % 
    %% VOTING
    % Winners are those alternatives that beat every other alternative in a \emph{head-to-head contest}.

    % We write \(n(x, y)\) for the number of agents who prefer alternative \(x\)
    % to alternative \(y\).
    % \vspace{1em}

    % A \emph{Condorcet winner} is an alternative \(x^*\) such that \(n(x^*, y) > n(x^*, y)\),
    % for any (other) alternative \(y\). 

    % Every voter \(i\) gives to alternative \(x\) a score of \(m - \textit{pos}_i(x)\), 
    % called \emph{the Borda score}, where \(\textit{pos}_i(x) \in \{1, \dots, m\}\)
    % is the position of \(x\) in \(i\)'s preference order \(\succ_i\).

    % The \emph{Borda winners} are the alternatives with the highest overall score, i.e., 
    % that maximize the sum of the Borda scores over all voters.



    %% APPORTIONMENT
    %% model
    % \begin{table}
    %     \begin{tabular}{rl}
    %         states                                  & \(N = \{1, \dots, n\}\)   \\
    %         population of state \(i\)               & \(p_i\)                   \\ 
    %         total population                        & \(p = p_1 + \dots + p_n\) \\
    %         number of seats to be allocated         & \(k\)                     \\
    %         seats allocated to state \(i\)          & \(k_i\)                   \\ 
    %         divisor                                 & \(d\)                      \\
    %         quota of state \(i\), for divisor \(d\) & \(\hat{q}_i = \nicefrac{p_i}{d}\) \\
    %         standard (true) quota of state \(i\)    & \(q_i = \nicefrac{p_i}{p}\cdot k\)\\ 
    %         upper quota of state \(i\)              & \(\lceil q_i \rceil\), i.e., \(q_i\) rounded up to the nearest integer \\
    %         lower quota of state \(i\)              & \(\lfloor q_i \rfloor\), i.e., \(q_i\) rounded down to the nearest integer
    %     \end{tabular}
    % \end{table}
    % 
    % \(\left\lfloor\frac{p_1}{d}\right\rfloor + \dots + \left\lfloor\frac{p_n}{d}\right\rfloor = k\)
    % \(
    %     \left[\frac{p_1}{d}\right] + \dots + \left[\frac{p_n}{d}\right] = k
    % \)
    % 
    % \(\left[\nicefrac{p_i}{d}\right]\)
    % 
    % \(s_1 = \left(\nicefrac{1}{10}, \nicefrac{9}{10}\right)\)
    % \(s_2 = \left(\nicefrac{1}{5}, \nicefrac{4}{5}\right)\)
    % \(\bm{s} = \left(s_1, s_2\right)\)
    % 
    % \begin{align*}
    %     u_1(\bm{s}) & = u_1(\text{Stag}, s_2)\cdot \Pr\left[\text{1 plays Stag}\right] + 
    %                     u_1(\text{Hare}, s_2)\cdot \Pr\left[\text{1 plays Hare}\right] \\
    %                 & = \bigg(u_1(\text{Stag}, \text{Stag})\cdot \Pr[\text{2 plays Stag}] + 
    %                      u_1(\text{Stag}, \text{Hare})\cdot \Pr[\text{2 plays Hare}]\bigg) \cdot \Pr\left[\text{1 plays Stag}\right] + \\ 
    %                 & ~~~~~~~~ \bigg(u_1(\text{Hare}, \text{Stag})\cdot \Pr[\text{2 plays Stag}] +
    %                      u_1(\text{Hare}, \text{Hare})\cdot \Pr[\text{2 plays Hare}]\bigg) \cdot \Pr\left[\text{1 plays Hare}\right] \\ 
    %                 & = (10 \cdot \nicefrac{1}{5} + 0 \cdot \nicefrac{4}{5}) \cdot \nicefrac{1}{10} +
    %                     (6 \cdot \nicefrac{1}{5} + 3 \cdot \nicefrac{4}{5}) \cdot \nicefrac{9}{10} \\ 
    %                 & = 3.44.
    % \end{align*}
    % 
    % \(\bm{s}^*= \big(\left(\nicefrac{1}{2}, \nicefrac{1}{2}\right), \left(\nicefrac{1}{2}, \nicefrac{1}{2}\right)\big)\)

    % \(
    %     f(x) = 
    %     \begin{cases}
    %         \lfloor x \rfloor, ~\text{if}~x < \sqrt{\lfloor x \rfloor \cdot \lceil x \rceil}, \\
    %         \lceil x \rceil, ~\text{if}~x \geq \sqrt{\lfloor x \rfloor \cdot \lceil x \rceil}.
    %     \end{cases}
    % \)

    % \(f{\left(\frac{p_1}{d}\right)} + \dots +f{\left(\frac{p_n}{d}\right)} = k.\)
    % \(f{\left(\nicefrac{p_i}{d} \right)}\)

    % A divisor method is the Huntington-Hill method if and only if
    % for all states \(i, j \in N\) such that \(\nicefrac{p_i}{k_i} \geq \nicefrac{p_j}{k_j}\),
    % it holds that:
    % \[
    %     \frac{\nicefrac{p_i}{k_i}}{\nicefrac{p_j}{k_j}} < \frac{\nicefrac{p_j}{(k_j-1)}}{\nicefrac{p_i}{(k_i+1)}}.
    % \]

    %% Condorcet Jury Theorem
    % \begin{table}
    %     \begin{tabular}{rl}
    %         voters & \(N = \{1, \dots , n\}\) \\
    %         alternatives & \(A = \{a,b\}\) \\
    %         correct alternative & \(a\) \\
    %         voter \(i\)'s vote & \(v_i \in A\) \\
    %         profile of votes & \(\bm{v} = (v_1, \dots , v_n)\) \\
    %         voter \(i\)'s competence & \(\Pr[v_i = a] = p_i\), with \(p_i \in [0,1]\) \\
    %         majority vote & \(F_{\maj}(\bm{v}) = x\), such that \(v_i = x\) for a (strict) majority of voters
    %     \end{tabular}
    % \end{table}

    % \begin{description}
    %     \item[(Competence)] Agents are better than random at being correct:
    %     \[
    %         p_i > \frac{1}{2},~\text{for any voter}~i \in N.
    %     \]
    %     \item[(Equal Competence)] All agents have the same competence:
    %     \[
    %         p_i = p_j = p,~\text{for all voters}~i,j \in N.
    %     \]
    %     \item[(Independence)] Voters vote independently of each other:
    %     \[
    %         \Pr[v_i = x, v_j=y] = \Pr[v_i=x] \cdot \Pr[v_j=y],~\text{for all voters}~i,j \in N.
    %     \]
    % \end{description}
    % \vspace{0.1em}

    % \({\Pr}{\left[F_\maj(v_1, \dots , v_n) = a\right]}\)
    % The profile is \(\bm{v} = (v_1)\).
    % The probability of a correct decision is:
    % \begin{align*}
    %     \Pr\left[F_\maj(v_1) = a\right] & = \Pr\left[v_1 = a\right] \\
    %                                     & = p \\
    %                                     & > \nicefrac{1}{2}.
    % \end{align*}
    % Note that as \(p\) grows, so does group accuracy.

    % Generalizing the previous identity we get the following recurrence:
    % {
    %     \footnotesize
    %     \[
    %         \textcolor{GameTheory}{{\Pr}{\left[ S_{n+2} > \left\lfloor\frac{n+2}{2} \right\rfloor \right]}} = 
    %         (1-p)^2 \cdot \textcolor{Azure}{{\Pr}{\left[ S_n > \left\lfloor\frac{n}{2}\right\rfloor + 1 \right]}} + 
    %         2p(1-p)^2 \cdot \textcolor{Honey}{{\Pr}{\left[ S_n > \left\lfloor \frac{n}{2}\right\rfloor \right]}} + 
    %         p^2 \cdot \textcolor{PastelGreen}{{\Pr}{\left[ S_n > \left\lfloor \frac{n}{2} \right\rfloor - 1 \right]}}        
    %     \]
    % }
    % The events on the right-hand-side can be rewritten as:
    % {
    % \footnotesize
    % \begin{align*}
    %     \textcolor{PastelGreen}{{\Pr}{\left[ S_n > \left\lfloor \frac{n}{2}\right\rfloor-1 \right]}} &= 
    %     \textcolor{Honey}{{\Pr}{\left[ S_n > \left\lfloor\frac{n}{2} \right\rfloor \right]}} + 
    %     \binom{n}{\lfloor\nicefrac{n}{2}\rfloor}\cdot p^{\lfloor\nicefrac{n}{2}\rfloor}(1-p)^{\lfloor\nicefrac{n}{2}\rfloor+1}\\
    %     \textcolor{Azure}{{\Pr}{\left[ S_n > \left\lfloor \frac{n}{2}\right\rfloor+1 \right]}} &= 
    %     \textcolor{Honey}{{\Pr}{\left[ S_n > \left\lfloor\frac{n}{2} \right\rfloor \right]}} -
    %     \binom{n}{\lfloor\nicefrac{n}{2}+1\rfloor}\cdot p^{\lfloor\nicefrac{n}{2}\rfloor+1}(1-p)^{\lfloor\nicefrac{n}{2}\rfloor}
    % \end{align*}
    % }
    % Plug the second and third identities into the first, 
    % and denote \(\binom{n}{\lfloor\nicefrac{n}{2}\rfloor} = \binom{n}{\lfloor\nicefrac{n}{2}\rfloor+1}=c\):
    % {
    %     \footnotesize
    %     \[
    %         \textcolor{GameTheory}{{\Pr}{\left[ S_{n+2} > \left\lfloor\frac{n+2}{2} \right\rfloor \right]}} = 
    %         \textcolor{Honey}{{\Pr}{\left[ S_n > \left\lfloor \frac{n}{2}\right\rfloor \right]}} + 
    %         c \cdot p^{\lfloor\nicefrac{n}{2}\rfloor+1}(1-p)^{\lfloor\nicefrac{n}{2}\rfloor+1}(2p-1).
    %     \]
    % }
    % Since \(\nicefrac{1}{2}<p<1\), the second term on the right-hand side is positive.

    If \(X_1\), \dots, \(X_n\) are independent and identically distributed (i.i.d.) random variables
    such that \(\EXP[X_i] = \mu\), then, for any \(\epsilon > 0\), it holds that:
    \[
        \lim_{n \rightarrow \infty} \Pr\left[ \left|\frac{X_1 + \dots + X_n}{n} - \mu\right|< \epsilon \right] = 1.
    \]
\end{document}